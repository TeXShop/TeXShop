About the Universal Version of OgreKit:

I slightly revised OgreKit so it will compile for both x86_64 and arm64. This short document explains how that was done. 

The OgreKit build settings for the various pieces had a "Valid Architectures" field containing one element, x86_64. Clicking this field brings up a small editor in which additional elements can be added. I added arm64 and arm64e. I don't know if the second if necessary, but I saw it in a WWDC slide. This field needs to be changed for each of the subprojects of OgreKit.

To build OgreKit, select "Build for Running". With this minor change, everything is ready to build, but the build won't work because the libonig.a linked library has only x86_64 code. Fixing this is somewhat tricky.  

You cannot find the key elements in the XCode directory on the left side. Instead, go to the full source code at OgreKit-3.0.2-Universal/RegularExpression/oniguruma. Inside this folder there is a subfolder named x86_64 with the source code. Make a copy of this folder and name the copy arm64. Inside this arm64 folder, type ./configure and then "make clean" and finally maybe ./configure again and then make. This will create the desired arm library in a hidden folder of the arm64 folder named .libs. 

Next make another subfolder of oniguruma named lipostuff. In both x86_64.libs and arm64.libs, find libonig.a and make copies of these files in lipostuff named libonigintel.a and libonigarm.a. Then run the command

	lipo libonigintel.a libonigarm.a -create -output libonig.a
	
You now have a universal library file. Test by running the command

	lipo -archs libonig.a
	
It should report binaries for x86_64 and arm64.

You are still now done. It turns out the build system will look for that library file in the x86_64 folder. So rename libonig.a in that folder to, say, libonigintel.a, and then copy libonig.a to the x86_64 folder.

But you still aren't done, because building OgreKit will remake the x86_64 copy of libonig.a. So the final step is to stop that from happening. In the folder OgreKit-3.0.2-Universal/RegularExpression/oniguruma, there is a file named Makefile. Rename this file MakeFile.original in case you need to go back to it. Then create a new (executible) Makefile containing essentially no instructions. The file I used is

-------
ONIGURUMA_VERSION=Onigmo-Onigmo-5.13.5
OS_VERSION=$(shell uname -r)

INTEL_64_DIR=x86_64


all: 

clean:
-----------

After all this, XCode will create a universal OgreKit framework and you are good to go.

Richard Koch
July 1, 2020

