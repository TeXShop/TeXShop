\documentclass[11pt, oneside]{amsart}
\usepackage{geometry}     
\geometry{letterpaper}         
\usepackage[parfill]{parskip} 
\usepackage{graphicx}
\usepackage{amssymb}
\usepackage{epstopdf}
\usepackage{url}

\title{The Sage Engine}
\author{Richard Koch}
\date{\today}

\begin{document}
\maketitle
\vspace{-.3in}
\section{Sage}
Sage is an open source alternative to Maple, Mathematica, and Matlab. The program provides all of the standard features expected from such a system: arbitrary precision arithmetic, symbolic integration and differentiation, two-dimensional plotting of functions, matrix algebra, and much more.
It can be downloaded from \url{https://www.sagemath.org}.
That site will in turn direct macOS users to a specific site with two install packages for Sage, one for Arm machines and one for Intel machines. We strongly recommend the latest version, currently SageMath 9.5. Sage is updated regularly, so by the time you read this document there may be a later version available. 


The Sage developers have dramatically improved installation for the Macintosh, so the many installation steps  we used to provide in this document can be replaced with the words ``follow their instructions.'' 

To use Sage, drag  the sage.engine file in this folder to $\sim/Library/TeXShop/Engines$, the folder of active engines for TeXShop. This step need only be done once, when you first install Sage. This engine will work with Sage 9.4 and higher.

%The Sage program comes with a folder of TeX style files named {\em sagetex}.  These style files allow Sage code to be embedded in a LaTeX document. During typesetting, Sage is called to process the Sage code, and the output is included in the pdf output file. This is an easy way to add graphs of functions to documents, perform symbolic integration, and include intricate numerical calculations. 
%
%The one tricky step is to find the sagetex folder in the Sage application  and copy it to a location where TeX will find it. Since sagetex depends strongly on the particular version of Sage being used, this step must be repeated each time Sage is updated on your system. 
%
%We will copy the sagetex folder to $\sim$/Library/texmf/tex/latex. Here $\sim$/Library is the Library folder in your home directory, not the system Library folder, and Apple usually hides the folder. But TeXShop has a command which opens
%$\sim$/Library/TeXShop. Create the texmf/tex/latex folders inside $\sim$/Library if they do not currently exist.


%% Sage and Latexmk

%In the folder containing the document you are now reading, there is a folder named ``Sage and Latexmk''  That folder contains the sagetex folder for SageMath-9-4. If you have version 9-4, drag and drop this sagetex folder to the location
%$\sim$/Library/texmf/tex/latex/ while holding down the Option key so you only make a copy of the folder.
%
%If you have a later version of SageMath, the program itself contains the sagetex folder for that version. Follow the steps below to obtain this folder. These instructions list the version number of Sage. Change the two occurrences of 9-4 to the later version number for future versions of Sage.
%\begin{verbatim}
%    cd /Applications/SageMath-9-4.app/Contents/Frameworks/Sage.framework
%    cd Versions/9.4/local/share/texmf/tex/latex
%\end{verbatim}
%
%These commands set the current Terminal directory to the location of sagetex. Issue the command
%\begin{verbatim}
%     open .
%\end{verbatim}
%
%to open this location in the Finder; notice the period at the end of this command. The result will be a Finder window containing the sagetex folder. Drag and drop this sagetex folder to the location
%$\sim$/Library/texmf/tex/latex/ while holding down the Option key so you only make a copy of the folder.



\section{How SageTeX Works}

This document ends with sample source code and output illustrating how sagetex works.
 In the source file, the initial line
\begin{verbatim}
     % !TEX TS-program = sage
\end{verbatim}
tells TeXShop to process the file using the sage engine; this engine first calls pdflatex, then calls sage, and finally calls pdflatex again.
The remaining lines in the preamble are standard LaTeX commands, except the required line
\begin{verbatim}
     \usepackage{sagetex}
\end{verbatim}
In the remaining source, sage commands are entered within lines of the form
\begin{verbatim}
     \sage{....}
\end{verbatim}
These lines cause sage to process commands and output LaTeX source fragments, which become part of the LaTeX document.

Notice in particular that sage can plot standard functions. Sage can also compute integrals symbolically; for example, look carefully at the command which processes $\int {{x^2 + x + 1} \over {(x - 1)^3 (x^2 + x + 2)}}$. This command contains standard LaTeX code to display the integral, but then Sage integrates and returns a typeset copy of the result.


%\section{Setting Up the Engine}
%
%This folder contains an engine file named ``sage.engine''. Move this file to the active portion of $\sim$/Library/TeXShop/Engines. Dan Drake, who is responsible for SageTeX, wrote this engine.
%
%The ``Sage'' program contains a style file named ``sagetex.sty'' and a number of support files. This style file is supposed to be copied to your TeX distribution. The file depends on other features of sage, so whenever you upgrade sage, you might also need to upgrade sagetex.sty in your TeX distribution. It is easy to forget to do this.
%
%But there is an ingenious way to work around this upgrade problem. Rather than copying the style file, we just
%create a symbolic link from it's location in TeX Live to its location in Sage. If we always rename the sage program to "SageMath", then this link will remain valid after updates and need not be changed. That is why we created a symbolic link inside TeX Live to Sage.

\section{Another Sample}

The ``Sage and Latexmk" folder contains a more extensive sample file called ``example.tex'' by Dan Drake. That sample is set up to use pdflatexmk, but it also works with the standard sage engine by changing the word ``pdflatexmk'' to ``sage'' on the top line.
Make a copy of this file in a separate folder and typeset to try out Sage.


\section{A Debugging Warning}

After one update to Sage, the sample document included in this folder stopped working. It turned out that the syntax for one Sage command had changed slightly. Breaking just one sage command caused them all to fail. Consequently, if you intend to use Sage together with TeX and suddenly nothing works, a little clever debugging will be required to determine and fix the bad Sage command.



\section{Final Remarks}
 
A Sage tutorial is available at the Sage page \url{https://www.sagemath.org}. It is definitely recommended. Extensive additional documentation is available at the same web page.
\end{document}

