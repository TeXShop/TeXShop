\documentclass[11pt, oneside]{amsart}
\usepackage{geometry}     
\geometry{letterpaper}         
\usepackage[parfill]{parskip} 
\usepackage{graphicx}
\usepackage{amssymb}
\usepackage{epstopdf}
\usepackage{url}

\title{The Sage Engine}
\author{Richard Koch}

\begin{document}
\maketitle
\vspace{-.3in}
\section{Sage}
Sage is a mathematical software system providing an open source alternative to Magma, Maple, Mathematica, and Matlab. It is available from 
\url{http://www.sagemath.org/} as a free download. The program provides all of the standard features expected from such a system: arbitrary precision arithmetic, symbolic integration and differentiation, two-dimensional plotting of functions, matrix algebra, and much more.

Download sage from the above web site. As of October, 2018, two Macintosh downloads are provided. One provides a regular Macintosh application, and the other provides a Unix command line interface. These instructions assume you obtained the version with a Macintosh application.

Move the resulting ``Sage'' program to /Applications. The
application will have a longer name, currently SageMath-8.4.app.  Long ago, the name of this application could be changed, and then for a time it couldn't be renamed. It seems that the name is again changeable.  We recommend changing the name to SageMath.app, because then the process described below won't have to be redone every time SageMath is updated.

(When renaming was first possible, we wrote 
\begin{verbatim}
If we keep the current name, the engine file will be tied to one particular
version of Sage. To fix this, change the name to just ``SageMath''. This 
clever step, which simplifies several steps below, is due to 
Daniel Grambihler.)
\end{verbatim}


SageTeX is a piece of the Sage download. It is basically a LaTeX style file, which  allows users to embed and process Sage code from within TeX files. The last three pages of this document show this style file in action. The source code on page four is followed by LaTeX output on pages five and six.

\section{How SageTeX Works}

 In the source file, the initial line
\begin{verbatim}
     % !TEX TS-program = sage
\end{verbatim}
tells TeXShop to process the file using the sage engine; this engine first calls pdflatex, then calls sage, and finally calls pdflatex again.
The remaining lines in the preamble are standard LaTeX commands, except the required line
\begin{verbatim}
     \usepackage{sagetex}
\end{verbatim}
In the remaining source, sage commands are entered within lines of the form
\begin{verbatim}
     \sage{....}
\end{verbatim}
These line cause sage to process commands and output LaTeX source fragments, which become part of the LaTeX document.

Notice in particular that sage can plot standard functions. Sage can also compute integrals symbolically; for example, look carefully at the command which processes $\int {{x^2 + x + 1} \over {(x - 1)^3 (x^2 + x + 2)}}$. This command contains standard LaTeX code to display the integral, but then Sage integrates and returns a typeset copy of the result.


\section{Setting Up the Engine}

This folder contains an engine file named ``sage.engine''. Move this file to the active portion of $\sim$/Library/TeXShop/Engines. This engine will work without changes if you rename the program to SageMath as recommended earlier.

 The fifth line from the bottom of this engine contains a full path to the sage binary inside the Sage program, and includes the precise name of the version of Sage downloaded. The current engine assumes that the program was renamed ``SageMath'' and uses that name.  In the initial engine, the line reads
\begin{verbatim}
/Applications/SageMath.app/Contents/Resources/sage/sage "$sagename"
\end{verbatim}
Here ``/Applications/SageMath.app'' is the name of the current SageMath being
used, and ``/Contents/Resoures/sage/sage'' reaches inside the application bundle to find the
sage binary it contains.
Dan Drake, who is responsible for SageTeX, wrote this engine.

The ``Sage'' program contains a style file named ``sagetex.sty'' and a number of support files. This style file is supposed to be copied to your TeX distribution. The file depends on other features of sage, so whenever you upgrade sage, you might also need to upgrade sagetex.sty in your TeX distribution. It is easy to forget to do this.

But there is an ingenious way to work around this upgrade problem. Rather than copying the style file, we just
create a symbolic link from it's location in TeX Live to its location in Sage. If we always rename the sage program to "SageMath", then this link will remain valid after updates and need not be changed.

Run Terminal in /Applications/Utilities. Type the following
\begin{verbatim}
     cd /usr/local/texlive/texmf-local/tex/latex
\end{verbatim}
Then type the following on a single line. Terminal may make a linefeed when
the line  grows too long and that is fine, but do not for instance make a linefeed
after the top "local" and before the bottom "/share".
\begin{verbatim}
     sudo ln -s /Applications/SageMath.app/Contents/Resources/sage/local
        /share/texmf/tex/latex/sagetex/sagetex.sty sagetex.sty
\end{verbatim}

Finally issue the command
\begin{verbatim}		
     sudo mktexlsr
\end{verbatim}
	



\section{Final Remarks}
 
A Sage tutorial is available at the Sage page \url{http://www.sagemath.org/help.html}. It is definitely recommended. Extensive additional documentation is available at the same web page.
\end{document}

