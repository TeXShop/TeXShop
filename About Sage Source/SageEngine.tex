\documentclass[11pt, oneside]{amsart}
\usepackage{geometry}     
\geometry{letterpaper}         
\usepackage[parfill]{parskip} 
\usepackage{graphicx}
\usepackage{amssymb}
\usepackage{epstopdf}
\usepackage{url}

\title{The Sage Engine}
\author{Richard Koch}
\date{\today}

\begin{document}
\maketitle
\vspace{-.3in}
\section{Sage}
Sage is a mathematical software system providing an open source alternative to Maple, Mathematica, and Matlab. Download from   
\url{https://www.sagemath.org/download.html} for free. The program provides all of the standard features expected from such a system: arbitrary precision arithmetic, symbolic integration and differentiation, two-dimensional plotting of functions, matrix algebra, and much more.

Begin by downloading Sage. Pick a repository, and then the page for Mac OS X, and then binaries for Intel, and finally a download package. Be sure to pick the package with name ending in x86\_64.app.dmg, rejecting the other offered distributions. 

Double click the dmg file to open it, and drag SageMath to your /Applications folder. Then rename the program to just SageMath.app. {\em It is very important to rename the program before running it for the first time.}

Using Apple's Terminal program in /Applications/Utilities, create a symbolic link to Sage in /usr/local/bin by typing each of the following lines. It is safest to  copy and paste these lines.
\begin{verbatim}
  cd /usr/local/bin
  sudo ln -s /Applications/SageMath.app/Contents/Resources/sage/sage .
\end{verbatim}

Next create a sagetex folder in /usr/local/texlive/texmf-local/tex/latex and make a symbolic link of sagetex.sty in that folder to allow TEX to use the sagetex package.
{\fontsize{8}{8}
\begin{verbatim}
 cd /usr/local/texlive/texmf-local/tex/latex 
 sudo mkdir sagetex
 cd sagetex
 sudo ln -s \
  /Applications/SageMath.app/Contents/Resources/sage/local/share/texmf/tex/latex/sagetex/sagetex.sty . 
 sudo mktexlsr
\end{verbatim}
}

Now run the SageMath program once; the first time it starts there will be a delay as it initializes various internal files. On Catalina, the first time you run the program, a dialog will appear warning that it is not properly signed. To get around this, hold down the command key and click on the program icon. A contextual menu will appear; click ``Open''. A warning dialog will appear, but continue anyway. From then on, the program will open naturally.

These steps are only needed when Sage is first installed. To update to a later version of Sage, download the program as before, move it to /Applications, rename it to SageMath.app, and run it once. Done.

\section{How SageTeX Works}

 In the source file, the initial line
\begin{verbatim}
     % !TEX TS-program = sage
\end{verbatim}
tells TeXShop to process the file using the sage engine; this engine first calls pdflatex, then calls sage, and finally calls pdflatex again.
The remaining lines in the preamble are standard LaTeX commands, except the required line
\begin{verbatim}
     \usepackage{sagetex}
\end{verbatim}
In the remaining source, sage commands are entered within lines of the form
\begin{verbatim}
     \sage{....}
\end{verbatim}
These line cause sage to process commands and output LaTeX source fragments, which become part of the LaTeX document.

Notice in particular that sage can plot standard functions. Sage can also compute integrals symbolically; for example, look carefully at the command which processes $\int {{x^2 + x + 1} \over {(x - 1)^3 (x^2 + x + 2)}}$. This command contains standard LaTeX code to display the integral, but then Sage integrates and returns a typeset copy of the result.


\section{Setting Up the Engine}

This folder contains an engine file named ``sage.engine''. Move this file to the active portion of $\sim$/Library/TeXShop/Engines. Dan Drake, who is responsible for SageTeX, wrote this engine.

The ``Sage'' program contains a style file named ``sagetex.sty'' and a number of support files. This style file is supposed to be copied to your TeX distribution. The file depends on other features of sage, so whenever you upgrade sage, you might also need to upgrade sagetex.sty in your TeX distribution. It is easy to forget to do this.

But there is an ingenious way to work around this upgrade problem. Rather than copying the style file, we just
create a symbolic link from it's location in TeX Live to its location in Sage. If we always rename the sage program to "SageMath", then this link will remain valid after updates and need not be changed. That is why we created a symbolic link inside TeX Live to Sage.

\section{A Debugging Warning}

After one update to Sage, the sample document included in this folder stopped working. It turned out that the syntax for one Sage command had changed slightly. Breaking just one sage command caused them all to fail. Consequently, if you intend to use Sage together with TeX and suddenly nothing works, a little clever debugging will be required to determine and fix the bad sage command.



\section{Final Remarks}
 
A Sage tutorial is available at the Sage page \url{https://www.sagemath.org/help.html}. It is definitely recommended. Extensive additional documentation is available at the same web page.
\end{document}

