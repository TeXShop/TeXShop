% !TEX TS-program = pdflatex


\documentclass[11pt, oneside]{article}   	% use "amsart" instead of "article" for AMSLaTeX format
\usepackage{geometry}                		% See geometry.pdf to learn the layout options. There are lots.
\geometry{letterpaper}                   		% ... or a4paper or a5paper or ... 
%\geometry{landscape}                		% Activate for rotated page geometry
\usepackage[parfill]{parskip}    		% Activate to begin paragraphs with an empty line rather than an indent
\usepackage{graphicx}				% Use pdf, png, jpg, or eps§ with pdflatex; use eps in DVI mode
								% TeX will automatically convert eps --> pdf in pdflatex		
\usepackage{amssymb}
\usepackage{url}
\usepackage{float}


\title{Preparing for the Demo}
\author{Richard Koch}
%\date{}							% Activate to display a given date or no date

\begin{document}
\maketitle

\section{Introduction}

This document is contained in the Demo folder you copied from $\sim$/Library/TeXShop/New. If you renamed that folder, the new name cannot have a space. Otherwise the demonstration will fail.

TeXShop 5 contains many new features. The new features are explained in detail in the pdf file ``TeXShop5-and-Interactive-Documents'', designed to be read in conjunction with a project called ``Fourier-for-TeXShop'' which uses the features.

As the demonstration proceeds, you will typeset portions of this project, until in the end you have both a pdf file and an html web page, together with support files for the web page. The pdf file is a static document, and the web page is an interactive document.  This pdf file and web page will work on any platform: Macintosh, Linux, Windows. 

But before you start the demonstration, you must make certain that your TeXShop distribution is up to date. There were changes in TeXShop 5.03, so make the changes below even if you already did similar things for the previous version.

\section{Updating Your Macintosh}

In TeXShop's ``TeXShop Menu'', select the item ``Open $\sim$/Library/TeXShop''. The Finder will open this location, showing several folders.  Open New/Version-5.03. This folder contains two engine files and two tex files. By holding down the Option key while dragging, drag copies of the two engine files to $\sim$/Library/TeXShop/Engines. This step may overwrite previous engines, but that is intended. Drag copies of both tex files to
$\sim$/Library/TeXShop/Templates. This may also overwrite previous files, and that is also intended.

The final step repeats a step for version 5.01, so skip if if you did it for 5.01.

The location $\sim$/Library/TeXShop/Movies/TeXShop may or may not contain a movie file named ``Getting Started.mp4". If it does not, go to the TeXShop Help menu and select the item ``TeXShop Demos/Getting Started". A short movie will play. A side effect of watching that movie is that the movie file was downloaded to the location in question.

Now go to $\sim$/Library/TeXShop/Movies/TeXShop and by holding down the Option key drag a copy of "Getting Started.mp4'' to ``$\sim$/Library/TeXShop/HTML.''

\section{Updating TeX Live}

TeX4ht is under very active development. I ran into one minor bug, but it was fixed the next day. If at all possible, update your copy  of TeX Live using TeX Live Utility on or after September 1, 2022. If not, one small problem may occur, but it will be explained in the main document at an appropriate time.


\section{OK, Start the Demo}

In the folder ``Fourier-for-TeXShop'' find the file Fourier.tex and open it in TeXShop. Open the file "Interactive-TeX4ht-Documents'' in your favorite pdf viewer. Start reading that document.




\end{document}