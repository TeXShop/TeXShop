\documentclass[11pt, oneside]{article}   	% use "amsart" instead of "article" for AMSLaTeX format
\usepackage{geometry}                		% See geometry.pdf to learn the layout options. There are lots.
\geometry{letterpaper}                   		% ... or a4paper or a5paper or ... 
%\geometry{landscape}                		% Activate for rotated page geometry
\usepackage[parfill]{parskip}    		% Activate to begin paragraphs with an empty line rather than an indent
\usepackage{graphicx}				% Use pdf, png, jpg, or eps§ with pdflatex; use eps in DVI mode
								% TeX will automatically convert eps --> pdf in pdflatex		
\usepackage{amssymb}

%SetFonts

%SetFonts


\title{LaTeX-dev Engines}
\author{Richard Koch}
%\date{}							% Activate to display a given date or no date

\begin{document}
\maketitle
%\section{}
%\subsection{}
At the national TUG meeting in Palo Alto on August 9, 2019, Frank Mittelbach talked about new procedures for the development of LaTeX. He explained that the  team developing modern versions of LaTeX has  a  testbed of documents which they test against before LaTeX releases. But occasionally, bugs are found after a release.

Therefore, the team has adopted a new release step. After a version of LaTeX passes internal tests, a ``dev'' version is released for the three major engines: PdfTeX, XeTeX, and LuaTeX. This version will appear in TeX Live. Users should occasionally test  their projects against this version, because if it passes these external tests, it will become the official release of LaTeX.

Consequently, three new engines are provided for TeXShop: LuaLaTeX-dev, PdfLatex-dev, and XeLaTeX-dev. They work exactly like LuaLaTeX, PdfLaTeX. and XeLaTeX except that they use the latest ``dev'' version of LaTeX. 


\end{document}  