\documentclass[11pt, oneside]{amsart}   	% use "amsart" instead of "article" for AMSLaTeX format
\usepackage{geometry}                		% See geometry.pdf to learn the layout options. There are lots.
\geometry{letterpaper}                   		% ... or a4paper or a5paper or ... 
%\geometry{landscape}                		% Activate for rotated page geometry
\usepackage[parfill]{parskip}    		% Activate to begin paragraphs with an empty line rather than an indent
\usepackage{graphicx}				% Use pdf, png, jpg, or eps§ with pdflatex; use eps in DVI mode
								% TeX will automatically convert eps --> pdf in pdflatex		
\usepackage{amssymb}

%SetFonts

%SetFonts


\title{ Readme - Arara}
\author{Richard Koch}
%\date{}							% Activate to display a given date or no date

\begin{document}
\maketitle
Arara is an alternative to latexmk, and to some extent an alternative to the engine mechanism in TeXShop.  To use arara, move the engine file from this folder to the main Engine folder.

The most recent version of Arara uses a precompiled library which is not developer signed. Therefore Arara cannot be included in MacTeX because MacTeX is notarized by Apple, and notarization requires that every piece of darwin code in the package have a developer signature. It is crucial that we notarize MacTeX because macOS will not install packages downloaded over the internet unless they are notarized. 

If you open TeX Live Utility in /Applications/TeX and click the {\em Packages} tab at the top, you will see a list of all possible TeX Live packages. If you obtained TeX Live by installing MacTeX--2024, the package ``arara'' will be marked ``forcibly-removed.'' 

But it is easy to get Arara using TeX Live Utility. Select the ``arara'' item in the package list provided by TLU and choose the menu item ``Install Selected Packages''. TeX Live Utility will install arara. TeX Live Utility and tlmgr use an internet delivery method which can deliver programs that are not notarized by Apple.





Note that arara uses Java. You may not have Java on your machine because Apple stopped installing it many years ago. To see if you have Java, open Apple's Terminal program and type
\begin{verbatim}
      java -version
\end{verbatim}
This will print the version of Java on your computer if you already have it, and otherwise explain how you can download it. Consult the Arara web site before downloading because recent reports suggest that Arara requires a specific version of Java.


\end{document}  