\documentclass[11pt, oneside]{article}   	% use "amsart" instead of "article" for AMSLaTeX format
\usepackage{geometry}                		% See geometry.pdf to learn the layout options. There are lots.
\geometry{letterpaper}                   		% ... or a4paper or a5paper or ... 
%\geometry{landscape}                		% Activate for rotated page geometry
\usepackage[parfill]{parskip}    		% Activate to begin paragraphs with an empty line\usepackage{graphicx}				% Use pdf, png, jpg, or eps§ with pdflatex; use eps in DVI mode
								% TeX will automatically convert eps --> pdf in pdflatex
								
%\usepackage{{Gratzer-Color-Scheme}	% Active to color theorems red and lemmas blue
	
\usepackage{amssymb}



\title{About the TeX4ht Engine}
\author{Richard Koch}
\date{\today}							% Activate to display a given date or no date

\begin{document}
\maketitle

TeX4ht is a program by Eitan M. Gurari which uses a standard LaTeX source file as
input, but outputs an html file. After Gurari died in 2009, other developers stepped up to
maintain the program, and it is in very active development today. The principal developer
is Michal Hoftich.

The file TeX4ht.engine is an engine file for this project which works in TeXShop 5.00 and
higher. To use it, drag TeX4ht.engine to the active area $\sim$/Library/TeXShop/Engines.

There has been a significant change in the TeX4ht engines starting with TeXShop 5.53.
The engines now assume that the source  directory contains a folder with a configuration file:
\begin{verbatim}
     MyConfig/myconfigfile.cfg
\end{verbatim}
The configuration
file is indicated by a flag in the make4ht command, and is read each time the typesetting
command runs. The reason for this change is that configuration files are the
recommended method to add features to TeX4ht, so  projects usually have them.

Using the engine is easy. Write a standard LaTeX source file. Select the TeX4ht engine in
the pulldown menu next to the typeset button and then push the Typeset button or type
command-T. The engine will call both pdflatex and TeX4ht, creating both a pdf output
file and an html output file. Then TeXShop will open the pdf file in a Preview window and
the html file in an HTML window. This makes it easy to compare the standard pdflatex
results with the corresponding web page. The web page will be active, so links and other
standard html features will work. You can modify the source and typeset again, and both
views will update.

By adding the following magic line to the top of the source file, the TeX4ht engine will
always be called and it is not necessary to use the pulldown menu:
\begin{verbatim}
% !TEX TS-program = TeX4ht
\end{verbatim}

\newpage

The folder containing this document  contains a short demonstration. Copy the folder
named “Example” to your home directory or location for TeX source files, and typeset the
document Sample.tex inside. Then compare the pdf and html outputs. Resize the windows
and notice their different behavior. Add extra material and typeset again to verify that
both views update. The sample file contains a web link. Click this in both the pdf and
html windows and notice their different behavior.

Math4ht has many parameters controlling its operation. Originally it created a large
number of small pictures for mathematical equations, and the web page displayed these
pictures. It can also output MathJax and let the browser interpret those commands. The
command to do that is given below, where “\$1” indicates a full path to the source file:
\begin{verbatim}
    make4ht -c MyConfig/myconfigfile.cfg "$1" "mathjax"
\end{verbatim}
TeX4ht can also output ordinary LaTeX code for mathematics and render the output
with MathML. The command for that is
\begin{verbatim}
     make4ht -c MyConfig/myconfigfile.cfg "$1" "mathml"
\end{verbatim}

This folder contains a sample configuration file, MyConfig/myconfigfile.cfg. It also
contains three engines for TeX4ht, which can use this configuration file.
The engines are
\begin{itemize}
\item TeX4ht.engine, which outputs MathJax data and displays both a pdf window and an html window
\item TeX4htMathML.engine, which outputs MathML data and displays both a pdf window and an html window
\item TeX4htOnly.engine, which outputs MathJax data and displays only an html window
\end{itemize}

In my experiments, MathJax was much better than MathML, so it is the recommended approach.
\end{document}  