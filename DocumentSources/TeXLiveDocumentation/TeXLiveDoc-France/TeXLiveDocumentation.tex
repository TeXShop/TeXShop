% !TEX encoding = UTF-8 Unicode 

\documentclass[11pt, oneside]{article}
\usepackage{geometry}            % See geometry.pdf to learn the layout options. There are lots.
\geometry{a4paper}           % ... or a4paper or a5paper or ... 
%\geometry{landscape}            % Activate for rotated page geometry
\usepackage[parfill]{parskip}    % To begin paragraphs with an empty line rather than an indent
\usepackage{graphicx}            % To include graphics.
\usepackage{amssymb}             % Common math support.
\usepackage[french]{babel}

\DeclareGraphicsRule{.tif}{png}{.png}{`convert #1 `dirname #1`/`basename #1 .tif`.png}
\usepackage[colorlinks=true, pdfstartview=FitV, linkcolor=blue, 
            citecolor=blue, urlcolor=blue]{hyperref}
            
%\usepackage{fontspec,xltxtra,xunicode} 
%\defaultfontfeatures{Mapping=tex-text} 
%\setromanfont{Hoefler Text} 
%\setromanfont{Optima}
%\setromanfont{Baskerville}
%\setsansfont[Scale=MatchLowercase] {Gill Sans} 
%\setmonofont[Scale=MatchLowercase] {Andale Mono} 

            
%\title{TeX Live Documentation}
%\author{The Author}
%\date{}                          % Activate to display a given date

\begin{document}
% \maketitle
% \tableofcontents
\pagestyle{empty}

\section{Documentation de \TeX\ Live}

\TeX\ Live est disponible avec une documentation détaillée :

\begin{itemize}
\item Une liste par ordre alphabétique de tout :
\url{file:///Library/TeX/Root/doc.html}.

\item Un aperçu de cette documentation :
\url{file:///Library/TeX/root/readme-html.dir/readme.en.html}.

\item Pour sélectionner d'autres langues dans lesquelles lire cet aperçu :
\url{file:///Library/TeX/root/index.html}.
\end{itemize}


Voici quelques exemples qui montrent la variété des informations disponibles :

\begin{itemize}

\item Un livre de 184 pages sur \LaTeX, \emph {The Not So Short Introduction to \LaTeX 2e}, de Tobias Oetiker, Hubert Partl, Irene Hyna et Elisabeth Schlegl : 
     \url{file:///Library/TeX/Root/texmf-dist/doc/latex/lshort-french/lshort-fr.pdf}
     
\item Recommandations des packages \LaTeX\ pour les tâches courantes : \url{file:///Library/TeX/Root/texmf-dist/doc/latex/latex-doc-ptr/latex-doc-ptr.pdf}

\item La documentation pour Beamer, une classe \LaTeX\ pour créer des présentations :
\url{file:///Library/TeX/Root/texmf-dist/doc/latex/beamer/beameruserguide.pdf}

\end{itemize}

\section{Pour démarrer avec \LaTeX}

Quelques liens sur le Net pour apprendre le \LaTeX\ :

\begin{itemize}
\item \url{https://tug.org/begin.html} - matériel d'introduction pour débuter en \TeX.

\item \url{https://learnlatex.org} - leçons de \LaTeX\ en ligne, avec exemples interactifs.

\item \url{https://tug.org/FontCatalogue} - polices disponibles, pour un usage avec \LaTeX.

\end{itemize}


\section{\TeX\ et TUG}

Quelques liens en ligne plus généraux :

\begin{itemize}
	
\item \url{https://tug.org} - Le site du groupe des utilisateurs de \TeX. Joignez-vous à eux ou faites un don si cela vous plaît !

\item \url{https://tug.org/whatis.html} - Une courte histoire de \TeX.

\item \url{https://tug.org/TUGboat} - tous les exemplaires de \textsl{TUGboat},
le principal journal à propos de  \TeX.

\item \url{https://tug.org/texlive} - page d'accueil de \TeX.

\item \url{https://ctan.org} - le Comprehensive \TeX\ Archive Network
(CTAN), avec une information détaillée sur tous les packages.

\end{itemize}

\end{document}
