% !TEX encoding = UTF-8 Unicode 

\documentclass[11pt, oneside]{article}
\usepackage{geometry}            % See geometry.pdf to learn the layout options. There are lots.
\geometry{a4paper}           % ... or letterpaper or a5paper or ... 
%\geometry{landscape}            % Activate for rotated page geometry
\usepackage[parfill]{parskip}    % To begin paragraphs with an empty line rather than an indent
\usepackage{graphicx}            % To include graphics.
\usepackage{amssymb}             % Common math support.

% Much better for texts in Spanish:
\usepackage{lmodern}
\usepackage[T1]{fontenc}
\usepackage[spanish]{babel}

\DeclareGraphicsRule{.tif}{png}{.png}{`convert #1 `dirname #1`/`basename #1 .tif`.png}
\usepackage[colorlinks=true, pdfstartview=FitV, linkcolor=blue, 
            citecolor=blue, urlcolor=blue]{hyperref}
            
%\usepackage{fontspec,xltxtra,xunicode} 
%\defaultfontfeatures{Mapping=tex-text} 
%\setromanfont{Hoefler Text} 
%\setromanfont{Optima}
%\setromanfont{Baskerville}
%\setsansfont[Scale=MatchLowercase] {Gill Sans} 
%\setmonofont[Scale=MatchLowercase] {Andale Mono} 

            
%\title{TeX Live Documentation}
%\author{The Author}
%\date{}                          % Activate to display a given date

\begin{document}
% \maketitle
% \tableofcontents
\pagestyle{empty}

\vspace*{-1.0cm}
\enlargethispage{2.0cm}

\section{Documentación de \TeX\ Live}

\TeX\ Live incluye una extensa documentación:

\begin{itemize}
\item Una lista alfabética de todo:
\url{file:///Library/TeX/Root/doc.html}.

\item Una descripción general de esta documentación:
\url{file:///Library/TeX/root/readme-html.dir/readme.en.html}.

\item Para seleccionar otros idiomas en los que leer la descripción general:
\url{file:///Library/TeX/root/index.html}.
\end{itemize}

Estos son algunos ejemplos que muestran la variedad de lo que hay disponible:

\begin{itemize}

\item Un libro de un centenar de páginas sobre \LaTeX, \emph {The Not So Short Introduction to
\LaTeX 2e}, por Tobias Oetiker et~al.:
\url{file:///Library/TeX/Root/texmf-dist/doc/latex/lshort-english/lshort.pdf}.

\item El libro anterior en español:
\url{file:///Library/TeX/Root/texmf-dist/doc/latex/lshort-spanish/lshort-a4.pdf}.
     
\item Paquetes de \LaTeX\ recomendados para tareas comunes: 
\url{file:///Library/TeX/Root/texmf-dist/doc/latex/latex-doc-ptr/latex-doc-ptr.pdf}

\item La documentación de Beamer, una clase de \LaTeX\ para crear presentaciones:
\url{file:///Library/TeX/Root/texmf-dist/doc/latex/beamer/beameruserguide.pdf}

\end{itemize}



\section{Empezando con \LaTeX}

Algunos enlaces online para aprender \LaTeX:

\begin{itemize}
\item \url{https://tug.org/begin.html} -- material introductorio para los que son nuevos en~\TeX.

\item \url{https://learnlatex.org} -- lecciones de \LaTeX\ en línea, con ejemplos interactivos.

\item \url{https://tug.org/FontCatalogue} -- tipografías disponibles y su uso en
\LaTeX.

\end{itemize}


\section{\TeX\ y TUG}

Algunos enlaces en línea más generales:

\begin{itemize}
	
\item \url{https://tug.org} -- la página web del \TeX\ Users Group. 
¡Por favor, únete o dona si lo deseas!

\item \url{https://tug.org/whatis.html} -- una breve historia de  \TeX.

\item \url{https://tug.org/TUGboat} -- todos los números \textsl{TUGboat},
la principal revista de~\TeX.

\item \url{https://tug.org/texlive} -- la página web de \TeX.

\item \url{https://ctan.org} -- la Comprehensive \TeX\ Archive Network
(CTAN), con amplia información sobre todos los paquetes.

\end{itemize}

\end{document}
