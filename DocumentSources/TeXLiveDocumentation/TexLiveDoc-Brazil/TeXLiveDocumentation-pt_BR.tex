% !TEX encoding = UTF-8 Unicode 

\documentclass[11pt, oneside]{article}
\usepackage{geometry}            % See geometry.pdf to learn the layout options. There are lots.
\geometry{letterpaper}           % ... or a4paper or a5paper or ... 
%\geometry{landscape}            % Activate for rotated page geometry
\usepackage[parfill]{parskip}    % To begin paragraphs with an empty line rather than an indent
\usepackage{graphicx}            % To include graphics.
\usepackage{amssymb}             % Common math support.

\DeclareGraphicsRule{.tif}{png}{.png}{`convert #1 `dirname #1`/`basename #1 .tif`.png}
\usepackage[colorlinks=true, pdfstartview=FitV, linkcolor=blue, 
            citecolor=blue, urlcolor=blue]{hyperref}
            
%\usepackage{fontspec,xltxtra,xunicode} 
%\defaultfontfeatures{Mapping=tex-text} 
%\setromanfont{Hoefler Text} 
%\setromanfont{Optima}
%\setromanfont{Baskerville}
%\setsansfont[Scale=MatchLowercase] {Gill Sans} 
%\setmonofont[Scale=MatchLowercase] {Andale Mono} 

            
%\title{TeX Live Documentation}
%\author{The Author}
%\date{}                          % Activate to display a given date

\begin{document}
% \maketitle
% \tableofcontents
\pagestyle{empty}

\section{Documentação do \TeX\ Live}

\TeX\ Live possui uma vasta documentação:

\begin{itemize}
\item Uma lista, ordenada alfabeticamente, com todos os pacotes:
\url{file:///Library/TeX/Root/doc.html}.

\item Um guia sobre essa documentação:
\url{file:///Library/TeX/root/readme-html.dir/readme.pt-br.html}.
\end{itemize}

Abaixo alguns exemplos para mostrar a variedade de documentação disponível:

\begin{itemize}

\item Um livro sobre \LaTeX\ com 139 páginas, \emph {The Not So Short Introduction to
\LaTeX 2e}, by Tobias Oetiker et~al.:
\url{file:///Library/TeX/Root/texmf-dist/doc/latex/lshort-english/lshort.pdf}
     
\item Recomendações de pacotes \LaTeX\ para tarefas comuns: \url{file:///Library/TeX/Root/texmf-dist/doc/latex/latex-doc-ptr/latex-doc-ptr.pdf}

\item Documentação sobre o Beamer, uma classe \LaTeX\ voltada para fazer apresentações de slides:
\url{file:///Library/TeX/Root/texmf-dist/doc/latex/beamer/beameruserguide.pdf}

\end{itemize}



\section{Iniciando com o \LaTeX}

Alguns materiais \textit{online} para aprender \LaTeX:

\begin{itemize}
\item \url{https://tug.org/begin.html} - material introdutório para quem não conhece o \TeX.

\item \url{https://learnlatex.org} - aulas de \LaTeX\ \textit{online} com exemplos interativos.

\item \url{https://tug.org/FontCatalogue} - catálogo de fontes disponíveis para serem usadas com o \LaTeX.

\end{itemize}


\section{\TeX\ and TUG}

Alguns outros materais mais gerais:

\begin{itemize}
	
\item \url{https://tug.org} - \textit{Site} do \textit{\TeX\ Users Group}. Por favor, participe ou faça doações, se desejar!

\item \url{https://tug.org/whatis.html} - uma história curta sobre o \TeX.

\item \url{https://tug.org/TUGboat} - todas as edições do \textsl{TUGboat}, a principal revista sobre \TeX.

\item \url{https://tug.org/texlive} - \textit{Site} do \TeX\ Live.

\item \url{https://ctan.org} - O \textit{Comprehensive \TeX\ Archive Network}
(CTAN) é um repositório com documentação vasta sobre todos os pacotes.

\end{itemize}

\end{document}
