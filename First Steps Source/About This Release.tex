
 \documentclass[11pt, oneside]{amsart}
\usepackage{multicol}
\usepackage{geometry}                % See geometry.pdf to learn the layout options. There are lots.
\geometry{letterpaper}                   % ... or a4paper or a5paper or ... 
%\geometry{landscape}                % Activate for for rotated page geometry
\usepackage[parfill]{parskip}    % Activate to begin paragraphs with an empty line rather than an indent
\usepackage{graphicx}
\usepackage{amssymb}
\usepackage{epstopdf}
\DeclareGraphicsRule{.tif}{png}{.png}{`convert #1 `dirname #1`/`basename #1 .tif`.png}

\usepackage[colorlinks=true, pdfstartview=FitV, linkcolor=blue, 
            citecolor=blue, urlcolor=blue]{hyperref}
            
\usepackage{url}


\title{About TeXShop 5.10}
\author{Richard Koch}
%\date{}                                           % Activate to display a given date or no date

\begin{document}
\maketitle
%\section{}
%\subsection{}
%  \begin{multicols}{2}

\thispagestyle{empty}
\vspace{-.3in}
\section{This Document}

If you are a new TeXShop user, you can skip this document and begin using the program.

The default behavior of TeXShop can be modified in two ways: by making Preference changes, and by editing files in $\sim$/Library/TeXShop. Here $\sim$/Library is the Library folder in your home directory. 

New versions of TeXShop generally do not change these user modifications because users don't like changes made behind their backs. But sometimes, new features require a few modifications. This paper explains what to do.
ccc.
In the future, please read this document when you upgrade. Each release of TeXShop will contain a new version.

\section{Changes in 5.04 - 5.10}

No changes in versions 5.04, 5.08, 5.09, or 5.10 require user interaction. Versions 5.05 through 5.07 were never released.

\section{Changes in 5.03}
Using an item in the main TeXShop menu, open ~/Library/TeXShop. You will see a list of folders. Open the folder New/5.03. It contains two templates and two engine files. Drag copies of the templates into the Templates folder. Drag copies of the engines into the Engines folder.  In both cases, you may be warned that you are overwriting earlier files. This rewrite is intensional. 


\section{Changes in 5.02}

It is important to make these changes even if you made the corresponding changes for version 5.01. Using an item in the main TeXShop menu, open ~/Library/TeXShop. You will see a list of folders. Open the folder New/5.02. It contains two templates and two engine files. Drag copies of the templates into the Templates folder. Drag copies of the engines into the Engines folder.  In both cases, you may be warned that you are overwriting earlier files. This rewrite is intensional. 

\section{Changes in 5.00 - 5.01}

Using an item in the main TeXShop menu, open ~/Library/TeXShop. You will see a list of folders. Open the folder New/5.01. It contains one template and two engine files. The template is named "TeX4ht-Interactive.tex"; drag a copy of this template into the Templates folder. The engines are named "html.engine" and "TeX4ht.engine"; drag copies of these two engines into the Engines folder. 

\section{Changes in 4.80-4.99}

These versions were never released.

\section{Changes in 4.73 - 4.79}

No changes in these versions require user interaction. 

\section{Changes in 4.72}

The latexmk engines in $\sim$/Library/TeXShop/Engines/Inactive/Latexmk were revised. We do not update active engines in $\sim$/Library/TeXShop/Engines, so if you have any active latexmk engines, replace them with revised copies. TeXShop ships with two such active engines, pdflatexmk.engine and sepdflatexmk.engine.

\section{Changes in 4.70}
Apple has been warning that the Python scripting language is deprecated and will eventually be removed from macOS.
It was finally removed in the latest update of macOS, Monterey 12.3, released in March, 2022. Luckily, TeXShop does not use Python. 

This caused developers to worry about other deprecated features of macOS. One of these is the default shell, which changed from bash to zsh in Catalina, introduced in 2019. Bash is still available in Monterey, but it has not been upgraded in many years and could be removed in the future. To protect against that, shell scripts in TeXShop that previously used bash have been changed to use zsh. In most cases, the switch was trivial.

Herbert Schulz revised the latexmk and transparency engines in TeXShop in this way. The new versions are in
$\sim$/Library/TeXShop/Engines/Inactive, in folders named ``GhostscriptTransparencyEngines'' and ``Latexmk''. The update to TeXShop 4.70 changed these but did not change your active engines, since you might have edited these engines. Most of these old engines  continue to work. Here is what Herbert Schulz responded when asked if users should update:

\begin{verbatim}
The basic engines, (pdf/xe/lua)latexmk, will work. The only engines that 
really will have to be updated are the (seldom used) pdftricksmk.engine 
and pst-pdfmk.engine since the 4 updated files in bin/tslatexmk have 
changes that need the updated engines.
\end{verbatim}

\begin{verbatim}
My general feeling is that all the engines that are used should be updated 
so I know the base I'm working with when things come up. Too bad changing 
to zsh from bash/sh wasn't as simple as changing the shebang line.
\end{verbatim}

\section{Changes in 4.69}

The file $\sim$/Library/TeXShop/bin/tslatexmk/latexmkrcDONTedit has been modified  to allow setting of a directory for most aux-like files using TeXShop's `parameter' directive. 
If you use pdflatexmk, you should move the present $\sim$/Library/TeXShop/bin/latexmkrcedit,  if it exists, to the desktop. Then restart TeXShop and typeset using any of the latexmk based engines. This recreates that file based on the DONTedit version. If you added any customizations to the original version of latexmkrcedit, copy them from the old version on the desktop to the new version.

New versions of the latexmk engines were provided in Engines/Inactive/latexmk in TeXShop 4.67 and 4.68. If you use one or more of these engines in the active engine area, but did not update to the new version recently, update now.


\section{Changes in 4.68}

No changes in this version require user interaction.  


\section{Changes in 4.67}

Herbert Schulz has carefully cleaned the pdflatexmk distribution, so users of this software should go to ~/Library/TeXShop/Engines/Inactive/Latexmk and read the documentation in the folder.

\section{Changes in 4.59 - 4.66}

No changes in these versions require user interaction.  

\section{Changes in 4.58}

The author of latexmk, John Collins, remarks that TeXShop engines used by latexmk should not move output files out of the source folder to alternate folders. This restriction can easily be enforced by adding the following lines to the file \url{~Library/TeXShop/bin/latexmkrcedit}:
\begin{verbatim}
     # make sure the output directories are not re-directed.
     $out_dir = $aux_dir = '';
\end{verbatim}

The latexmkrcedit file just mentioned is created automatically by latexmk the first time it is used, and then left in place so some users can edit it and add personal features. If you have never edited this file, please remove it from
\url{~Library/TeXShop/bin}. A revised version will be written the first time latexmk is used. If you {\em have} edited latexmkrcedit, please re-edit the file now and add the lines above.

\section{Changes in 4.52 - 4.57}

Version 4.52 was never released. No changes in 4.53 - 4.57 require user interaction.  


\section{Changes in 4.50 and 4.51}

Versions 4.45 through 4.49 were never released. 

In  $\sim$/Library/Engines/Inactive/GhostscriptTransparencyEngines/For TeXShop/Engines, find the file  latexTR.engine. This file or a copy should be moved to $\sim$/Library/Engines to become one of the active engines. See the Changes document for an explanation.

\section{Changes in 4.44}

The default TeXShop Coloring Theme for Dark Mode has been modified in version 4.44. An easy way to obtain this theme is to open the folder $\sim$/Library/TeXShop/Themes, perhaps using a TeXShop menu item to open $\sim$/Library.  Then quit TeXShop. Rename the file DarkTheme.plist to something else, like OldDarkTheme.plist. Finally restart TeXShop. It will recreate DarkTheme.plist with new values, but you still have the old theme if you turn out to like it better.

\section{Changes in 4.43}

No changes in 4.43 require user interaction.

\section{Changes in 4.42}

No changes in 4.42 require user interaction.

\section{Changes in 4.41}

No changes in 4.41 require user interaction.


\section{Changes in 4.40}

In TeXShop Preferences under the Typesetting tab, the top left item selects the default typesetting engine. The available choices are TeX, LaTeX, or one of the typesetting engines defined in the active area of $\sim$/Library/TeXShop/Engines and listed in the source pull-down menu next to the Typesetting button.  ConTeXt used to be listed as another choice, but since there is an engine for ConTeXt, the separate ConTeXt choice has been removed. Most user choices will be preserved after the change, but a check after updating is recommended.


\section{Changes in 4.39}

No changes in 4.39 require user interaction.

\section{Changes in 4.38}

No changes in 4.38 require user interaction.

\section{Changes in 4.37}

Version 4.36 was never released. No changes in 4.37 require user interaction.


\section{Changes in 4.35}

Versions 4.32 through 4.34 were never released. No changes in 4.35 require user interaction.

\section{Changes in 4.31}

Versions 4.28 through 4.30 were never released. There are a few changes that require user interaction in Version 4.31. 

Go to
$\sim$/Library/TeXShop; notice that TeXShop has a menu command to take you there. Find the folder
$\sim$/Library/TeXShop/New/Version-4.31/CommandCompletion. Drag the file inside this folder to $\sim$/Library/TeXShop/CommandCompletion.

Find the folder $\sim$/Library/TeXShop/New/Version-4.31/Macros. If you have never edited the Macros file from TeXShop, then drag the file Macros\_Latex.plist from this location to $\sim$/Library/TeXShop/Macros, overwriting the existing file in that folder. Otherwise, copy the file XML-Macros.plist in $\sim$/Library/TeXShop/New/Version-4.31/Macros to the desktop. Open the Macro editor in TeXShop, select ``Add macros from file...'', navigate to the desktop file, and add the two macros to your macro list. (Note: to open the Macro Editor, you must use the Macros Menu in the TeXShop menu list. The item ``Add macros from file...'' will be in this menu, but only after you open the Macros Editor.)

Find the folder $\sim$/Library/TeXShop/New/Version-4.31/Stationery. Drag the two files inside this folder to $\sim$/Library/TeXShop/Stationery.

The final step is not recommended. Find the folder $\sim$/Library/TeXShop/New/Version-4.31/Themes.
This folder contains new versions of the Themes shipped with TeXShop; the new versions contain colors for the five extra xml syntax coloring operations. You could drag copies of these themes to $\sim$/Library/TeXShop/Themes. However, you certainly should not do this if you changed any theme colors. Moreover, adding the extra colors is totally unnecessary because TeXShop will automatically provide default xml syntax colors, and if you edit any of these defaults, then the new values will be added to the appropriate themes file. However, you might like to drag just PreTeXt-dark.plist to $\sim$/Library/TeXShop/Themes, because it is a new theme simulating colors used at the PreTeXt conference in Portland.





\section{Changes in 4.09 - 4.27}

No changes in these versions require user intervention. But it is crucial to read the Changes document for versions 4.08 through 4.22 since there are many Mojave-related changes. Since 4.22, changes have been minor.

\section{Changes in 4.08}

Only two changes in 4.08 may require user intervention.

The first concerns missing tools on window toolbars. Users may discover that some of their tools on the Source and Preview window toolbars are missing. This problem will occur only on Mojave, and only if an older version of TeXShop ran first, probably before an update. To fix this problem, open a project with both a Source window and a Preview window. With the Source window active, select ``Customize Toolbar'' in the TeXShop Window menu, and drag the default tools to the Source toolbar. With the Preview window active, select ``Customize Toolbar'' and drag the default tools to the Preview toolbar. Then with the Source Window active, select ``Use One Window'' in the TeXShop Window menu. With this new window active, again select ``Customize Toolbar'' and drag the default tools to the Single Window toolbar. After these actions, all tools should appear.

In previous versions of TeXShop, syntax coloring of the source file was turned off during file loading, and the user had to jiggle the mouse to start this coloring. This code may have been present to fix a bug created by trying to color too early, but experiments suggest that the bug is not longer present. So in version 4.08, source windows are colored immediately.

If this causes problems when loading large documents, the new behavior can be turned off by setting a hidden preference using Apple's Terminal program:
\begin{verbatim}
     defaults write TeXShop ColorImmediately NO
\end{verbatim}




\section{Changes in 3.99 - 4.07}

No changes in 3.99 - 4.01 require user intervention. Some users may want to take this moment to change their encoding preference to UTF-8 Unicode.

\section{Changes in 3.98}

At some point in the past, a submenu was added to the Macros menu called ``Magic Lines'' listing all special comment lines (i.e., Magic Line comments) except program, encoding, and root. Each item inserts the comment into the source code.

But a later version of TeXShop inadvertently omitted this item. It is now back. If you never add your own macros and don't have the Magic Line macros, quit TeXShop, go to $\sim$Library/TeXShop and remove the Macros folder, and then restart TeXShop. A new Macros folder will be generated with the latest macros.

Otherwise, go to $\sim$Library/TeXShop/New/Macros, open the item Magic-Line-Macros.plist using the Macro Editor, load the new Macros, and position them as desired.



\section{Changes in 3.90 - 3.97}

Version 3.90 and 3.93 were never released. No changes in 3.91 - 3.92, 3.94 - 3.97 require user intervention.


\section{Changes in 3.89}

Version 3.89 introduces one more ``magic line'' for source files, and consequently one new macro to insert this magic line. The magic line is mainly for ConTeXt users, allowing them to revert to the 2016 version of the synctex parser (the authors of ConTeXt wrote their own synctex support without taking note of changes made in 2017 by the synctex author Jerome Laurens). 

If you have never edited TeXShop macros to add your own macros, then it is easy to obtain the new macros.
Go to the  folder \path{~/Library/TeXShop} in your home directory. This folder is hidden, but there is a command in TeXShop to take you there. Move the Macros folder in this location to the desktop,
and then quit TeXShop. When TeXShop is next started, it will create a new Macros folder containing all of the latest macros.

If you have edited the default Macros, it is easy to add the new macro. Open the Macro editor. Create a new macro titled ``Use Old Sync Parser (ConTeXt)'' with content
\begin{verbatim}
     % !TEX useOldSyncParser
\end{verbatim}

Herbert Schulz has revised the latexmk engines in TeXShop to use the magic line ``Extra Parameter for Engine Call.'' This allows users to add  flags to calls to TeX binary programs when latexmk calls these binaries, without rewriting the engine. The revised engines are in \path{~/Library/TeXShop/Engines/Inactive/Latexmk}. Replace any active latexmk engines in
\path{~/Library/TeXShop/Engines} with new versions from the Inactive folder.

\section{Changes in 3.88}

Version 3.88 contains new macros to insert ``magic lines'' in source files. So users no longer need to remember the syntax of these lines.

If you have never edited TeXShop macros or added your own macros, then it is easy to obtain the new macros.
Go to the Library folder in your home directory and find $\sim$/Library/TeXShop. This folder is hidden, but there is a command in TeXShop to take you there. Move the Macros folder in this location to the desktop,
and then quit TeXShop. When TeXShop is next started, it will create a new Macros folder containing all of the latest macros.

If you have edited Macros, you can follow the previous paragraph making sure to save the old Macros folder on the desktop. Then use the TeXShop Macros editor to add these old macros to the new ones, and rearrange your macros, throwing away duplicates. But there is an easier way. In $\sim$/Library/TeXShop/New/Macros there is a file named ``Magic-Line-Macros.plist." Open this file in the Macro editor and add these macros to your existing macros.


\section{Changes in 3.87}

No changes in 3.87 require user intervention.

\section{Changes in 3.86}

Markdown is a very simple markup language by John Gruber. Files written in this language can easily be converted to web pages, pdf files, \LaTeX\ files, and many other formats. The $\sim$/Library/TeXShop/Engines/Inactive folder has a few files which may interest users of Markdown. The folder is called pandoc because pandoc is one possible engine which can convert between many different Markup languages. 

The first item in this folder is Pandoc.pdf, a pdf document with links to Gruber's article and to the Pandoc page, where a simple free install package for pandoc on OS X can be obtained. This Pandoc page has extensive information on conversions which Pandoc can do.

The pandoc folder also contains Markdown.md and Markdown.comment, which can be moved to the Stationery folder to provide Markdown stationery. Users can edit the stationery and add more useful examples.

The folder also contains two engines, Md2pdf.engine and Md2HTML.engine. The first converts Markdown code to pdf and opens the pdf file in TeXShop. The second converts Markdown to HTML and opens the html file in Safari.
Further extensions for other conversions could easily be written.

\section{Changes in 3.78 - 3.85}

Versions 3.78 and 3.79 were never released. No changes in 3.80 - 3.85 require user intervention.



\section{Changes in 3.76 - 3.77}

Version 3.76 was never released. No changes in it, or 3.77, require user intervention.


\section{Changes in 3.60 - 3.75}

No changes require user intervention.

\section{Changes in 3.59}

Only one change requires  action by users. The {\em Encoding} macro now sets the default encoding to IsoLatin9 rather than MacOSRoman. To make this change, 
select ``Open Macro Editor'' in the Macros menu. Select ``Encoding'' in the left column of the editor. Sixteen lines from the top of the source on the right, find the line reading
\begin{verbatim}
     property default_encoding: "MacOSRoman"
\end{verbatim}
Change this to read
\begin{verbatim}
     property default_encoding: "IsoLatin9"
\end{verbatim}
and push the {\em Save Button} at the bottom of the editor. Some users may wish to select a different default.


\section{Changes in 3.55, 3.56, 3.57, and 3.58}

No changes require user intervention.

\section{Changes in 3.54}

Sparkle was broken in 3.53. It is fixed in 3.54. This is the only change.

\section{Changes in 3.53}

Near the end of the Macro menu, there is an item ``Tables $\rightarrow$ Table'' which inserts the
outline for a table into the source. The first line of this source is 
\begin{verbatim}
     \begin{table}[htdp]
\end{verbatim}

However, ``d'' is not a legal parameter for tables. Previous versions of LaTeX ignored this error, but it is flagged as an error in the 2015 version. 

The easiest way to fix this is to open the TeXShop Macro editor and remove the ``d''. If you have never edited the default macros,
you can also get the fix by quitting TeXShop, going to $\sim$/Library/TeXShop and
throwing the entire Macros folder in the trash. Then restart TeXShop and the current
Macros folder will be created.


\section{Changes in 3.52}


This release is primarily about preparing for the release of OS X 10.11, El Capitan. In El Capitan, the TeX binaries are found at the symbolic link /Library/TeX/texbin rather than at /usr/texbin.

This requires changes in engine files, which often reset the \$PATH variable. We add {\em both}
/usr/texbin and /Library/TeX/texbin to \$PATH to make engines comparible with old and new systems. You will automatically receive updates to the Inactive Engines in $\sim$/Library/TeXShop/Engines/Inactive. But active engines in $\sim$/Library/TeXShop/Engines must be modified
by you.

If you never touch these Engine files, the easy way to modify them is to quit TeXShop, drag the entire Engine folder to the trash, and restart the program. A new Engine folder will be created. Otherwise,  edit the active engine files using TeXShop. Modify the path statement near the top of the file, if one exists. The syntax of this line depends on the shell being used, but will be obvious by looking at the current line being modified. Add /Library/TeX/texbin to the list.

Release 3.52 also modifies a few macros which explicitely mention /usr/texbin. In these cases, the reference must be changed to /Library/TeX/texbin. The modified macros are not automatically
installed in $\sim$/Library/TeXShop/Macros.  I'll explain  how to obtain them in the
following paragraphs.

It is safe to make the changes on current systems {\em provided MacTeX-2015 or BasicTeX-2015}
has been installed on these systems, because then the link /Library/TeX/texbin was created by
the 2015 packages. This holds even if you  switched back to TeX Live 2014 with the
TeX Dist Preference Pane. But if you didn't install MacTeX-2015 or BasicTeX-2015, wait until you do so to modify your macros.

If you never edit macros, the easy way to obtain new ones is to quit TeXShop, throw the folder
$\sim$/Library/TeXShop/Macros in the trash, and then restart TeXShop. The Macros folder will be recreated with the new Macros.

Otherwise, edit the file Macros\_Latex.plist directly by opening it in TeXShop. (Do not double
click the file to open it; instead drag it to the TeXShop icon in the dock or use TeXShop's Open menu.)  Find all occurrences of /usr/texbin; there should be three of them.
Change each of them to /Library/TeX/texbin.





\section{Changes in 3.51}

TeXShop has new macros by Michael Sharpe which make it easier to add and edit tables. To
examine these Macros, go to $\sim$/Library/TeXShop/New/Macros and copy the items tabularize.plist and tabularize.pdf to the desktop. The second is documentation for the macros.

To add the Macros, select ``Open Macro Editor'' in the Macro menu. Then select ``Add Macros
from file...'', which appears in this menu. Navigate to the desktop and select the file
tabularize.plist.


\section{Changes in 3.40, 3.41 through 3.50}

Version 3.40 was never released. Versions 3.41 through 3.50 require no user actions.

\section{Changes in 3.39}

The most significant change in 3.39 is the addition of Michael Sharpe's ``Recent TeX Fonts'' document and associated font Macros.  Michael and I attended the TeX User Group meeting in Portland, Oregon at the end of July, 2014. I knew him as an Applescript expert; several of his scripts are in TeXShop. Other macros from him will appear in a future version. To see his scripts, go to 
\url{https://dl.dropboxusercontent.com/u/3825336/TeX/index.html}. Pay particular attention to 
{\em macrocopier.zip} on this location, a stand alone program which makes it easy to maintain and extend TeXShop macros.

At the TUG meeting, I discovered that Sharpe is a font expert widely known to users on other platforms.
TeX Live contains a very large number of \TeX\ fonts, but it is not that easy to use them.
Most font sets don't have mathematical symbols, and it becomes a design task to find pleasing combinations of fonts for text, sans-serif sections, and mathematics.

Sharpe wrote a document called {\em Recent TeX Fonts}, now available in the TeXShop Help menu. This document describes a number of pleasing font combinations, one per page. Each page lists the features of a set, provides an extensive sample of text and mathematics typeset using
it, and contains the exact La\TeX\ code needed to use the font set. These sets are the result
of extensive work by Sharpe; I understand that some of them took four months to perfect.

One way to use the document is to select an article or book written with standard fonts and  copy Sharpe's implementation section into the document's header and retypeset. 

To make this even easier, Sharpe slightly modified the LaTeX template which comes with TeXShop, 
defining a  section in the header bounded by {\em \%SetFonts} comments. The space between two such comments can be empty when the document is originally written.  Sharpe defined three
macros called ``GetFontSet,'' ``SaveFontSet,'' and ``TestFontSet.'' The first of these brings up
a small dialog listing known font sets. When one is selected, its implementation code is written to the source document between the {\em \%SetFonts} comments, replacing other implementation code there. So with one click and one typeset, the document can be seen written with a new set of fonts.

Users can also put their own implementation code between the  {\em \%SetFonts} comments.
The SaveFontSet macro reaches between the comments and saves the implementation code to a file in \path{~/Library/TeXShop/bin} named {\em SetFonts}, which is used by the GetFontSet macro to list known font combinations. Thus {\em SetFonts}  gradually builds into a library of known font combinations.

I'll let the TestFontSet macro speak for itself.

To use these Sharpe additions, it is necessary to use the full TeX Live as installed by MacTeX,
because BasicTeX doesn't contain many fonts. All the font sets defined by Sharpe have been
tested and are available with TeX Live 2014, except two. The garamondx font has a license permitting free personal use provided the font is not sold. This font is on CTAN, but it cannot
be in TeX Live because TUG sells a DVD containing TeX Live. However, a script named getnonfreefonts is available to download and install this font. See \url{https://www.tug.org/fonts/getnonfreefonts/}.

The {\em SetFonts} template also has a Lucida entry. Lucida is a commercial font, sold by TUG and others. 
See \url{https://www.tug.org/store/lucida/index.html}. Many users own this set, and Sharpe's detailed and non-trivial code to use them will help them obtain the most from the fonts.

It is our hope that the existence of these easy techniques will lead to more \TeX\ documents that don't scream out ``I was written with \TeX,'' and instead have a professional printed look.

To use these new features, some customization of TeXShop is required. (The customization isn't
necessary if you are running TeXShop for the first time.) To perform the steps, use the
Finder's ``Go'' menu to go to $\sim$/Library/TeXShop, a folder which contains many subfolders.
\begin{itemize}
\item The TeXShop folder contains a subfolder named New, which has a subfolder named Templates. This folder contains one template: LaTeXTemplate.tex. Copy it to the
active $\sim$/Library/TeXShop/Templates folder. The LaTeXTemplate.tex folder will replace an existing template. If, of course, you edited this template, then you won't want to overwrite it and instead you'll need to merge the {\em \%SetFonts} comments from the new LaTeXTemplate.tx into your customized template.
\item Inside New there is another folder named Macros/FontSets. This folder contains five items. 
The ``refreshfront'' item can be ignored, since a copy has already been placed in $\sim$/Library/TeXShop/bin. The ``bg8.pdf'' file needs to be placed in $\sim$/Library/texmf/tex/latex, and you may have to create these folders inside $\sim$/Library. Finally, the three files GetFontset.plist, SaveFontSet.plist, and TestFontSet.plist contain the new macros, which need to be loaded into
the Macros menu.
\item If you have never modified the Macros which come from TeXShop, then quit TeXShop and
move the entire Macros folder at $\sim$/Library/TeXShop/Macros to the trash. When you restart TeXShop, a  new Macros folder
will be created containing the old macros and the three new macros, and you are done.

But if you  edited the macros, then run TeXShop and select Open Macro Editor in the Macros menu. In the TeXShop  Macros menu,   select the item {\em Add macros from file.} Navigate to 
one of GetFontset.plist, SaveFontSet.plist, or TestFontSet.plist and select it.  A new macro will appear. Drag it to an appropriate spot, perhaps just under ``Open quickly..." Repeat for the other two macro files, and then save the changes.
\end{itemize}



\section{Changes in 3.36, 3.36.1,  3.36.2, 3.37, and 3.38}

No user actions are required for these versions.

\section{Changes in 3.35}

TeXShop 3.35 is compiled on Mavericks. Due to bugs in Lion, it was necessary to disable magnification in the Preview window and selection of rectangles in the Preview window on that system. If at all possible, Lion users should upgrade to Mountain Lion or Mavericks, where these features are restored. Lion users who cannot
upgrade might consider using version 3.26 of TeXShop, permanently available on the TeXShop web page. Magnification and rectangle selection work on that version (but the necessary code does not compile on Mavericks). Version 3.26 will not be upgraded.

TeXShop 3.35 contains a GotoLabel macro by Micael Sharpe. New users will obtain it automatically, but older users need to go to \textasciitilde/Library/TeXShop/New/Macros
and find the macro in a folder named GotoLabel, which also has a  README explaining how to install it.



\section{Changes in 3.32, 3.33, 3.34}

Versions 3.27 - 3.31 were never released. For version 3.32, TeXShop source code was converted to ARC and compiled with Automatic Reference Counting
turned on. Thus memory management is done automatically by Apple's compiler.

Version 3.34 has new LilyPond and MetaPost engines by Nicola Vitacolonna.  Active versions of the metapost engines are included in TeXShop. To update
the active  engines, go to  $\sim$/Library/TeXShop/Engines and replace the current engines named ``nv-metafun.engine'' and ``nv-metapost.engine'' by the new
versions of these engines in 

\hspace{.5in} $\sim$/Library/TeXShop/Engines/Inactive/MetapostEngines/Engines.

\section{Changes in 3.26}

Four Applescript macros have problems with recent versions of OS X:  Open Quickly, Insert Reference, New Array, and New Tabular. These  were fixed by
Michael Sharpe. If you use 
default Applescript Macros and haven't edited them or added your own Macros, you can obtain
the new versions by quitting TeXshop, going to $\sim$/Library/TeXShop and moving the Macros
folder to the desktop, and then restarting TeXShop. A new version of the Macros file will be
created. If all is well, the old Macros folder can be thrown away.

Otherwise, find the files InsertReference-OpenQuickly.plist and NewTabular-NewArray.plist
in $\sim$/Library/TeXShop/New. In the Macros menu select ``Open Macro Editor''. Then
select ``Add macros from file...'' and add the macros from both of these files. The
new macros will appear at the bottom of the Macro List. Use the Macro Editor to
replace the old contents of the four macros with the new contents using Copy and Paste.
Finally throw the new copies at the end of the Macro list away.

 
\section{Changes in 3.25} 

The TeXShop web site is \url{http://pages.uoregon.edu/koch/texshop/texshop.html}, but used to be \url{http://www.uoregon.edu/~koch/texshop/texshop.html}. This is changed throughout TeXShop. Users need to edit one of the TeXShop macros to make the change there. Open the Macros editor and edit the 
``Applescript/TeXShop home page'' macro to point to the modern link.
 
\section{Changes in 3.24}

No user actions are required for this version. Just one bug is fixed. The Sparkle
upgrade mechanism failed on Mavericks for many version up to and including 3.23. This
is fixed.

\section{Changes in 3.22 and 3.23}

TeXShop 3.23 fixed one bug. In 3.22, if the user configured TeXShop to use an external editor
and then set the hidden preference ExternalEditorTypesetAtStart to YES,  the program crashed when opening a file. This is fixed in 3.23.

No user actions are required for version 3.22 unless you use the Sage engine. If you use Sage, go to $\sim$/Library/TeXShop/Engines/Inactive/Sage, read ``About Sage'', make the required changes, and switch to the new Engine file.

\section{Changes in 3.21}

Versions 3.19 and 3.20 were never released. No user actions are required for version 3.21.

Users with Portables connected to large monitors may want to make one change. For a  long time we have recommended that users new to
TeXShop  arrange the location and size of the Source and Preview windows on the desktop as desired. Most users prefer a side by side configuration with the Source window on the left and the Preview window on the right. Then activate the source
window and in the Source menu select ``Save Source Position''. Similarly activate the preview window and in the Preview menu select ``Save Preview Position''. From that point on, all
TeXShop source and preview windows will appear in the selected positions.

If you have a portable connected to a large monitor, this configuration works as long as you are attached to the monitor. But when you are traveling, the windows will appear on the screen of your portable, and probably not in ideal positions. To fix this, arrange a source and preview window on your portable screen; the portable can be attached to the large monitor at the time. Then activate the source window and in the Source menu select ``Save Source Position for Portable''. Activate the preview window and in the Preview menu select ``Save Preview Position for Portable''. After this step, windows will appear in the desired position when you are connected to the large monitor at home or office, and windows will appear in the desired position on the portable screen when you are traveling.


\section{Changes in 3.18}

User action is only required if SyncTeX is incorrectly configured. Users should
check that
\begin{itemize}
\item in TeXShop Preferences under the Typesetting tab, the ``Sync Method'' is set to SyncTeX;
\item in TeXShop Preferences under the Engine tab, the two configuration lines for ``pdfTeX''
each contain the following flags
\begin{verbatim}
     --file-line-error --syncTeX=1
\end{verbatim}
\item in TeXShop Preferences on the same page, the two ``TeX + dvips + distiller'' lines
contain the following instruction
\begin{verbatim}
     --extratexopts "-file-line-error -synctex=1"
\end{verbatim}
	The easy way to do this is to push the four ``Default'' buttons beside these four entries.
\end{itemize}

\section{Changes in 3.17}

The latexmk engines have been modified. By default, $\sim$/Library/TeXShop/Engines contains two such engines, pdflatexmk.engine and sepdflatexmk.engine. Drag the new versions of these engines,
which are in $\sim$/Library/TeXShop/Engines/Inactive/Latexmk, into this location.


\section{Changes in 3.16}

No changes require user action.

\section{Changes in 3.15}

No changes require user action {\em unless} you use Monoco as an editing font. Apple has optimized the
text and font rendering in Mountain Lion for the Retina display. On a non-Retina display running on Mountain Lion, some fonts in small sizes will appear slightly blurry. To switch back to the old behavior, issue the following
command in Terminal:
\begin{verbatim}
defaults write TeXShop NSFontDefaultScreenFontSubstitutionEnabled -bool YES
\end{verbatim}

\section{Changes in 3.12 - 3.14}

Versions 3.12 and 3.13 were never released. If you have already made changes for 3.11,
the only change required for 3.14 is to move the files {\em nv-metafun.engine} and {\em nv-metapost.engine} from $\sim$/Library/TeXShop/New/Engines to
$\sim$/Library/TeXShop/Engines, replacing old versions of these engines. The new metapost engines are by Nicola Vitacolonna. Documentation
about them is in $\sim$/Library/TeXShop/Engines/Inactive/MetaPostEngines-1.4.4.

\section{Changes in 3.11}

The ConTeXt engines have been renamed. This is the only change. I promised to make this change a year ago, but checking MacTeX-2012 shortly
before release, I found that the promise was ignored. The old and new names are
\begin{itemize}
\item ConTeXt-MKIV.engine $\rightarrow$ ConTeXt (LuaTeX).engine
\item ConTeXt-xetex.engine $\rightarrow$ ConTeXt (XeTeX).engine
\item ConTeXt .engine $\rightarrow$ ConTeXt (pdfTeX).engine
\end{itemize}
The new names make explicit the TeX program which will run the ConTeXt macros for that engine.

The new files were placed in  $\sim$/Library/TeXShop/Engines/Inactive folder. To complete the change,
remove the old files from $\sim$/Library/TeXShop/Engines/Inactive/ConTeXt, and rename
 ConTeXt-MKiv.engine to ConTeXt (LuaTeX).engine  in $\sim$/Library/TeXShop/Engines.

\section{Changes in 3.10}

Two important changes in TeXShop 3.10 require user intervention. You may want to read this entire section before making the changes, but here are instructions when you are ready:
\begin{itemize}
\item Go to TeXShop Preferences and select the Engine tab. Press both ``Default'' buttons in the pdfTeX section of the pane in the middle of the left side.
\item Go to the Library folder in your home directory. This folder is hidden in Lion, but you can get to it from the Finder's ``Go'' menu by pressing the option key while the menu is selected.  In this Library folder, there is a TeXShop folder containing several subfolders. Inside TeXShop/New/Templates is a template named
``LatexTemplate.tex''. Move this to TeXShop/Templates, overwriting the old template with the same name. If you edited this old template, you'll need to merge
your changes with the changes from the new version.
\end{itemize}

The key feature of TeXShop 3.10 is removal of the ``--shell-escape'' flag for the pdftex and pdflatex engines. This flag gives \TeX\  permission to
call other programs during typesetting. Unfortunately, the flag allows {\em any} shell program to be run, including for instance a short command which erases everything in your home folder. So when the flag is set, you could download a \TeX\ file from a malicious web site, typeset it, and lose all of your files.

Pushing the two ``Default'' buttons in Preferences removes the flags.

Why was this flag activated in previous versions of TeXShop? A primary reason is that many users have old documents containing eps illustrations. While pdfTeX
and pdfLaTeX can accept illustrations in many formats, including pdf, jpg, and png, they cannot accept eps illustrations. But the {\em epstopdf} package can call  Ghostscript during typesetting and automatically convert eps illustrations to pdf illustrations. The flag gave pdfTeX permission to call Ghostscript.

Two years ago, the TeX Live distribution made this easier and safer. A new ``restricted shell escape'' command was added to \TeX\ giving \TeX\ permission to call a carefully limited list of programs during typesetting, including Ghostscript. Then the {\em graphicx} package was modified to automatically convert eps illustrations to pdf format during typesetting, without explicitly including epstopdf. This means that old projects with eps illustrations will automatically typeset with pdflatex, provided the source includes the {\em graphix} package. 

At the time of these changes, we did not remove ``--shell-escape'' because it was still needed to convert tiff files to png format automatically during typesetting. This conversion is done by {\em /usr/bin/convert} , a program from ImageMagick which is installed as part of MacTeX. Unfortunately, TeX Live could not add convert to the restricted
shell escape list because in the Windows world there is a program with this name which can do dangerous things. Also we knew that tex4ht, which can be used to typeset \TeX\ source and output html pages, calls convert.

Recently, we discovered that while the tex4ht {\em script} calls convert, the convert program is not called by the pdflatex portion of the script. So the flag is not needed for
tex4ht. 

TeXShop 3.10 contains a new command in the File menu called ``Convert tiff.'' This item is active when a TeX source window is active. Selecting the item brings up a  dialog listing all files and folders in the
directory containing the source file. Only tiff files are active in this dialog; others are grayed out. Multiple items can be chosen in the dialog. Pushing the  ``Convert''
button at the bottom of the dialog will create png versions of all selected tif or tiff illustrations. 

The {\em Convert tiff} dialog could be strengthened in the future to provide other graphic conversions, if there is sufficient demand.

The new Latex Template removes the inclusion of epstopdf because it is no longer necessary, and removes  code to automatically convert tiff files to png using {\em convert} because it will not run without ``--shell-escape'' and has been replaced with the new TeXShop menu command.




There is one final change in TeXShop 3.10. For many years, TeXShop has provided two different ways to create projects governed by a master root file with several include files. The earlier method used a menu command {\em Create Project Root}. The later more powerful method involves adding ''\% !TEX root = ...'' to the
first few lines of included files. This is the preferred method.

We have  removed the menu item {\em Create Project Root}. Old projects using this mechanism will still typeset correctly, but new projects should use the  
''\% !TEX root = ...'' syntax instead.

\section{Changes in 3.07, 3.08, and 3.09}

The pdflatexmk engine is now active by default for new users. Older users can activate it by finding ``pdflatexmk.engine'' in $\sim$/Library/TeXShop/Engines/Inactive/Latexmk
and drag it to $\sim$/Library/TeXShop/Engines. No other 3.07, 3.08, or 3.09 changes require user intervention.

\section{Changes in 3.05 and 3.06}

In 3.05, the folder $\sim$/Library/TeXShop/CommandCompletion  has a new subfolder named GratzerMathCC. This folder contains
an optional CommandCompletion.txt file with additional completions by George Gratzer, useful when typesetting mathematics.
It also contains documentation by Gratzer and Herbert Shultz about these extensions.

To obtain the folder, move it from $\sim$/Library/TeXShop/New/CommandCompletion to $\sim$/Library/TeXShop/CommandCompletion.
To activate the new completions, move CommandCompletion.txt from the new folder up a level to the main CommandCompletion
folder, replacing the similarly named file there now.


No additional changes were made in 3.06.

\section{Changes in 3.00 - 3.04}

Version 3 requires Lion. It is the start of a new sequence of TeXShop Lion releases. 

For some time to come, the old 2.** series of TeXShop releases for Tiger, Leopard, and Snow Leopard and the new 3.** releases for Lion
will be developed in parallel. The Sparkle update
system will only upgrade TeXShop 2.** to later 2.** releases and will only upgrade TeXShop 3.** to later 3.** releases. It will never upgrade, say, TeXShop 2.43 to
TeXShop 3.02.

Versions 3.00 and 3.01 were renamed TeXShop-64. This broke several applescript macros, and turns out to be unnecessary for separating the 2.** and 3.** releases in Sparkle. So version 3.02 has the old TeXShop name. 

For a detailed list of Lion features, see the Lion section of the TeXShop Help Panel, available under the TeXShop Help menu. The Internationalized versions of
this Panel are often out of date. Changes are also described at \url{http://pages.uoregon.edu/koch/texshop} under the Lion tab.

\section{Changes in 2.42 and 2.43}
There are revised engines in $\sim$/Library/TeXShop/Engines/Inactive for ConTeXt running on top of LuaTeX, XeTeX, and pdfTeX.
In particular, the ConTeXt (LuaTeX) engine is now a default engine, replacing ConTeXt-MKIV in previous versions of TeXShop. Drag this file
from $\sim$/Library/TeXShop/Engines/Inactive/ConTeXt to $\sim$/Library/TeXShop/Engines.



\section{Changes in 2.41}

The only change is an updated Command Completion.pdf document. Find this document in
\begin{quotation}
 $\sim$/Library/TeXShop/New/CommandCompletion
 \end{quotation}  and move it to  \begin{quotation}$\sim$/Library/TeXShop/CommandCompletion\end{quotation}

\section{Changes in 2.40}

If you already have 2.39, no changes are needed for 2.40.

\section{Changes in 2.39}

One new item is provided for 
	$\sim$/Library/TeXShop/CommandCompletion: a French translation of Herbert Schulz's documentation  named
	Completement2011.pdf.
Find this item in
\begin{quotation} 
$\sim$/Library/TeXShop/New/CommandCompletion
\end{quotation}
 and move it to 
 \begin{quotation}
 $\sim$/Library/TeXShop/CommandCompletion
 \end{quotation}


\section{Changes in 2.38}

Three new items are provided for $\sim$/Library/TeXShop/CommandCompletion: a revised document {\em Command Completion for TeXShop.pdf}
and two folders named IndentedCC and Quick Start Guide for Command Completion. Find these three items in
\begin{quotation} 
$\sim$/Library/TeXShop/New/CommandCompletion
\end{quotation}
 and move them to 
 \begin{quotation}
 $\sim$/Library/TeXShop/CommandCompletion
 \end{quotation}

The documentation in $\sim$/Library/TeXShop/Engines/Inactive has been revised, but the revised version is installed automatically, so no action
is needed when upgrading. According to the revised documentation, users of the ConTeXT-MKIV.engine must run the following
command ONCE in Terminal before the engine will work:
\begin{verbatim}
     luatools --generate
\end{verbatim}


\section{Changes in 2.35, 2.36, and 2.37}


If you already have 2.36, no changes are needed for 2.37.

If you already have 2.34 and are upgrading to 2.37, read this section and ignore everything else. The only significant change is that there is a new active engine: ConTeXt-MKIV. This is the version of ConTeXt running on top of LuaTeX.

Find the ConTeXt-MKIV.engine in $\sim$/Library/TeXShop/Engines/Inactive/ConTeXt. Drag or copy it to $\sim$/Library/TeXShop/Engines. Done.


\section{Changes in 2.34}

If you already have 2.33 and are upgrading to 2.34, read this section and ignore everything else. The only significant change is that there is a new active engine: LuaLaTeX. That is because LuaLaTeX, under development for several years, has reached the stage in TeX Live 2010 when it can be used for serious work. TeX Live 2010 will be released shortly.

Find LuaLaTeX.engine in $\sim$/Library/TeXShop/Engines/Inactive/LuaTeX. Drag or copy it to $\sim$/Library/TeXShop/Engines. Done.

\section{Preference Changes;  No New Changes After 2.30}

TeXShop 2.30 can find errors in projects governed by a root document. When the user selects ``Goto Error,'' the source file containing the error is opened and the line with the error is highlighted. TeXShop finds errors by parsing the console output. In the default operation of TeX, the console error message does not identify the source file, but this can be changed if TeX is run with the ``file-line-error'' flag. Therefore, the following Preference changes are needed under the Engine tab. If you modified TeXShop's defaults earlier, for instance by omitting the shell-escape flag, modify the suggestions below appropriately.

Under the engine tab, the pdfTeX item should be
\begin{verbatim}
     pdftex --file-line-error --shell-escape --synctex=1
\end{verbatim}
the pdfLaTeX item should be
\begin{verbatim}
     pdflatex --file-line-error --shell-escape --synctex=1 
\end{verbatim}
the TeX item for TeX + dvips + distiller should be
\begin{verbatim}
     simpdftex etex --maxpfb --extratexopts "-file-line-error -synctex=1"
\end{verbatim}
and the corresponding Latex item should be
\begin{verbatim}
     simpdftex latex --maxpfb --extratexopts "-file-line-error -synctex=1"
\end{verbatim}

\section{A Refresher Course; No Action Required}

TeXShop creates a folder $\sim$/Library/TeXShop containing several subfolders. One of these folders is named ``Templates''; it  contains templates for various kinds of TeX documents. I'll use that folder as an example, although it has not changed recently. Users can edit these templates, add templates of their own, and throw away inappropriate templates. TeXShop displays these templates in a pulldown menu on the source toolbar; users select a template to insert its source code in their document.

When TeXShop is upgraded, there might be new versions of the default templates, but it would certainly be inappropriate to reach into the user's carefully edited Templates folder and change its contents. Therefore, TeXShop upgrades install the new templates in $\sim$/TeXShop/New.  Users can examine them at their leisure and activate ones they like by moving those to $\sim$/TeXShop/Templates.

For the record, there is a way to obtain the exact contents of $\sim$/Library/Templates as seen by a brand new user.  To get it, quit TeXShop and move the entire Templates folder to the desktop. Then restart TeXShop. When it discovers that the Templates folder is completely missing, TeXShop replaces it with a new default copy. The same mechanism works with any subfolder of $\sim$/Library/TeXShop. However, this drastic action should almost never be necessary because of changes introduced in TeXShop 2.33.

\section{A Slight Upgrade Change From 2.33 Onward}

In the past, upgrades did not modify any folder in $\sim$/Library/TeXShop. Starting with TeXShop 2.33, three folders are touched:  bin, Engines/Inactive, and Scripts. Extra files added to those folders by users are left unchanged, but default TeXShop files are replaced by new versions and new default TeXShop files are added. Note that active Engine files are not changed because they don't live in the Inactives folder.

These changes make upgrading engines much easier. Support files are automatically upgraded, so users only need to upgrade by hand  the actual engine files, which seldom change.

\section{Easy Steps for Some Users}

Some TeXShop users have never edited files in $\sim$/Library/TeXShop, except perhaps to modify the default Templates, have never added new Macros to the Macro menu, have never added  new Engines, and  have never used Command Completion or at least never added words to the Completion Dictionary.
These users can complete the upgrade easily. Quit TeXShop, open $\sim$/Library/TeXShop, and move the following three folders to the desktop:
CommandCompletion, Macros, Keyboard. Then restart TeXShop. Done. But it might be nice to keep the old desktop copies of these folders for a few days in case you made a modification you had forgotten, and find that TeXShop's behavior has changed.

\section{New And Improved Engines}

In version 2.30, Nicola Vitacolonna made beautiful new engines for metapost and metafun,  To obtain these engines, move nv-metafun.engine and nv-metapost.engine from 
\begin{verbatim}
     ~/Library/TeXShop/Engines/Inactive/Metapost
\end{verbatim}
to $\sim$/Library/TeXShop/Engines.

The XeTeX and XeLaTeX engines have been modified to contain the file-line-error flag. To obtain these new versions, move XeTeX.engine and XeLaTeX.engine from
\begin{verbatim}
     ~/Library/TeXShop/Engines/Inactive/XeTeX
\end{verbatim}
to $\sim$/Library/TeXShop/Engines, replacing the older versions there now.

In 2.31 there is a new Sage engine by Dan Drake. If you used the old Sage engine, you will need to switch to the new one because SageTeX is now included in Sage. Because there are several changes in SageTeX, you need to read ``About Sage'' in 
\begin{verbatim}
     ~/Library/TeXShop/Engines/Inactive/Sage
\end{verbatim}
for important details.

The latexmk engines maintained by Herbert Schulz are upgraded regularly. Schulz has modified these engines and their support files so that in the future upgrades will occur automatically without user action. But users must make one change in 2.32 to switch to the new files; if this change is not made, the old latexmk from TeXShop 2.30 will stay in place and continue working. 

Changing is easy for most users. Latexmk comes with six engine files, located in 
\begin{verbatim}
     ~/Library/TeXShop/Engines/Inactive/Latexmk
\end{verbatim}
Drag new versions of those which you use from this location to $\sim$/Library/TeXShop/Engines, replacing the older versions there now. Done.

A small number of users may have edited support files  for latexmk that used to be in $\sim$/Library/TeXShop/bin. These edited versions will remain unchanged in this location, but new support files are now provided in $\sim$/Library/TeXShop/bin/tslatexmk. In the new latexmk, these support files should not be edited because they will be upgraded by TeXShop upgrades. Instead, Schulz has provided a mechanism to add personal changes to a new editable file. Read the documentation in 
\begin{verbatim}
     ~/Library/TeXShop/Engines/Inactive/Latexmk
\end{verbatim}
to see how this is done. It is only necessary to take action if you edited the previous support files.

\section{Macros; No New Changes After 2.30}

Alan Munn provided a wonderful new macro named ``Paste Spreadsheet Cells.'' Using his macro, you can copy cells from a spreadsheet and paste these cells, embedded in appropriate TeX code, into your source. To obtain the macro, go to
\begin{verbatim}
     ~/Library/TeXShop/New/Macros
\end{verbatim}
and copy the file PasteSpreadsheetCells.plist to the desktop. Then open TeXShop and in the Macro menu select ``Open Macro Editor.'' Select the ``Add macros from file...'' item in this menu, navigate to the desktop copy of PasteSpreadsheetCells.plist, and choose it. A new ``Paste Spreadsheet Cells'' macro will be added to your Macro list. If you desire, drag it to a different spot in the list, and then hit the Save button.

\section{Keyboard Shortcuts; No New Changes After 2.31}

TeXShop has the ability to remap Keyboard Shortcuts. This feature was activated by only a few users, and stopped working some time ago because the file controlling it contained comments within comments, which is illegal in xml. If you modified Keyboard Shortcuts in the past, copy the file
\begin{verbatim}
     ~/Library/TeXShop/Menus/KeyEquivalents.plist
\end{verbatim}
 to  the desktop. Then in all cases find the file
\begin{verbatim}
     ~/Library/TeXShop/New/Menus/KeyEquivalents.plist
\end{verbatim}
and move it to the folder
\begin{verbatim}
     ~/Library/TeXShop/Menus
\end{verbatim}
overwriting the old file. The new file is only an template explaining how to make changes, but the changes it makes are commented out. In the unlikely event that you edited the old KeyEquivalents.plist, merge in your changes from the desktop copy.


\section{Command Completion Changes In 2.31}

TeXShop has Command Completion. Type the beginning of a command and hit the Escape key. TeXShop will complete the command. If several completions are possible, hit Escape several times to cycle between them.
The list of known completions is stored in $\sim$/Library/TeXShop/CommandCompletion and  can be edited within TeXShop.

This facility has been expanded by Herbert Schulz in version 2.30. To use his additions, you need a new CommandCompletion file. 
If you modified the default Command Completions sometime in the past, copy the file 
\begin{verbatim}
     ~/Library/TeXShop/CommandCompletion/CommandCompletion.txt 
\end{verbatim}
 to  the desktop. Then in all cases find the file
 \begin{verbatim}
     ~/Library/TeXShop/New/CommandCompletion/CommandCompletion.txt 
\end{verbatim}
and move it to the folder  
\begin{verbatim}
     ~/Library/TeXShop/CommandCompletion 
\end{verbatim}
overwriting the old file.  If you modified the old file, you must edit CommandCompletions.txt with TeXShop or TextEdit and merge in your changes from the desktop copy.

\section{Documentation Changes In 2.30}

The ``Paste Spreadsheet Cells'' macro by Alan Munn is documented in TeXShop Help under Macros Help, Default Applescript Macros.

Herbert Schulz's extensions to Command Completion are explained in a short paper he wrote, which can be found in $\sim$/Library/TeXShop/New/CommandCompletion.

Nicola Vitacolonna's new engines for MetaPost, nv-metafun and nv-metapost, are explained in his ReadMe in
$\sim$/Library/TeXShop/Engines/Inactive/MetaPost. This folder also contains a folder of examples.



% \end{multicols}

\end{document}  