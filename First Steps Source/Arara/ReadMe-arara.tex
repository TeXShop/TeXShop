
 \documentclass[11pt, oneside]{amsart}
\usepackage{multicol}
\usepackage{geometry}                % See geometry.pdf to learn the layout options. There are lots.
\geometry{letterpaper}                   % ... or a4paper or a5paper or ... 
%\geometry{landscape}                % Activate for for rotated page geometry
\usepackage[parfill]{parskip}    % Activate to begin paragraphs with an empty line rather than an indent
\usepackage{graphicx}
\usepackage{amssymb}
\usepackage{epstopdf}
\DeclareGraphicsRule{.tif}{png}{.png}{`convert #1 `dirname #1`/`basename #1 .tif`.png}

\usepackage[colorlinks=true, pdfstartview=FitV, linkcolor=blue, 
            citecolor=blue, urlcolor=blue]{hyperref}
            
\usepackage{url}


\title{ReadMe - Arara}
\author{Richard Koch}
%\date{}                                           % Activate to display a given date or no date

\begin{document}
\maketitle
%\section{}
%\subsection{}
%  \begin{multicols}{2}

\thispagestyle{empty}
\vspace{-.3in}

Arara is an alternative to latexmk, and to some extent an alternative to the engine mechanism in TeXShop. The manual is in this folder. To use arara,  move the engine file from this
folder to the main Engine folder.

The arara program is part of TeX Live, so you can ignore the sections in the manual about installing  and compiling it. Note that arara uses Java. You might not have Java on your machine because Apple removes it when security bugs are discovered in it. To see if you have Java, type
\begin{verbatim}
     	java -version
\end{verbatim} 
in /Applications/Utilities/Terminal. This will print the version of Java on your computer if you already have it, and otherwise bring up Oracle's Java web page, where you can download it,
 \end{document}  