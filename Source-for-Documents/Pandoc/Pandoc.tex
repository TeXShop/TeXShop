\documentclass[11pt, oneside]{article}   	% use "amsart" instead of "article" for AMSLaTeX format
\usepackage{geometry}                		% See geometry.pdf to learn the layout options. There are lots.
\geometry{letterpaper}                   		% ... or a4paper or a5paper or ... 
%\geometry{landscape}                		% Activate for rotated page geometry
\usepackage[parfill]{parskip}    		% Activate to begin paragraphs with an empty line rather than an indent
\usepackage{graphicx}				% Use pdf, png, jpg, or eps§ with pdflatex; use eps in DVI mode
								% TeX will automatically convert eps --> pdf in pdflatex		
\usepackage{amssymb}
\usepackage{url}

%SetFonts

%SetFonts


\title{Pandoc Engines}
\author{Richard Koch}
%\date{}							% Activate to display a given date or no date

\begin{document}
\maketitle
Pandoc is an open source program capable of converting files to and from a large number of markup formats.
It is available at \url{http://pandoc.org}. Among the markups it understands are Markdown, LaTeX, pdf, HTML, docx, rtf, and many more.

The above web site has an install package for Mac OS X, which runs on Sierra and earlier systems. This is a straightforward way to install. (Other methods listed, including one that installs a complete Haskell system, are recommended for experts only).

Markdown is an  easy markup language by John Gruber, which simplifies the construction of web pages.
See \url{https://daringfireball.net/projects/markdown/syntax} and the Wikipedia article on Markdown. 
Recently a TeXShop user told me that a survey was constructed of editors used to write Markdown and TeXShop landed in the middle of the pack. This led me to find out more about the program.

This folder contains two engines for use with Markdown, and a piece of stationery and corresponding comment file for using it. Put the stationery file and comment file in the TeXShop/Stationery folder, and the engines in the active Engine area of TeXShop/Engines.

One engine converts a Markdown (.md) file to HTML, and then opens the HTML in Safari. The other converts it to pdf, where it is opened by TeXShop itself. Notice that Markdown to pdf allows embedded LaTeX code; see the Pandoc documentation for details.

Many other programs are available on the web to convert Markdown to HTML. For almost all of them, construction of a corresponding engine is very easy.


\end{document}  