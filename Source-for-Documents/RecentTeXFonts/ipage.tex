\documentclass{article}
\usepackage[margin=1.18in]{geometry} 
\usepackage[parfill]{parskip}% Begin paragraphs with an empty line, no indent
\usepackage[parfill]{parskip}% Begin paragraphs with an empty line rather than an indent
\usepackage{booktabs}
\usepackage{graphicx}
\usepackage{xcolor}
\usepackage{lipsum}
\usepackage{microtype}
%SetFonts
% libertine+newtxmath
\usepackage[full]{textcomp}
\usepackage[sups,osf]{fbb}
\usepackage[T1]{fontenc}
\usepackage{cabin}
\usepackage[varqu,varl]{zi4}% inconsolata
\usepackage{amsmath,amsthm}
\usepackage[libertine,bigdelims,vvarbb]{newtxmath}
\usepackage[cal=boondoxo]{mathalfa}
\pagestyle{empty}
\def\Pr{\ensuremath{\mathbb{P}}}
\def\d{\mathrm{d}}
\begin{document}
\thispagestyle{empty}
\lipsum[1]
The typeset math below follows the ISO recommendations that only variables
be set in italic. Note the use of upright shapes for $\d$, $\mathrm{e}$
and $\uppi$. (The first two are entered as \verb|\mathrm{d}| and
\verb|\mathrm{e}|, and in fonts derived from {\tt mtpro2} or {\tt newtxmath},
 the latter is entered as \verb|\uppi|.)

\textbf{Simplest form of the \textit{Central Limit Theorem}:} \textit{Let
$X_1$, $X_2,\cdots$ be a sequence of iid random variables with mean $0$ 
and variance $1$ on a probability space $(\Omega,\mathcal{F},\Pr)$. Then}
\[\Pr\left(\frac{X_1+\cdots+X_n}{\sqrt{n}}\le y\right)\to\mathfrak{N}(y)\coloneq
\int_{-\infty}^y \frac{\mathrm{e}^{-t^2/2}}{\sqrt{2\uppi}}\,
\mathrm{d}t\quad\mbox{as $n\to\infty$,}\]
\textit{or, equivalently, letting} $S_n\coloneq\sum_1^n X_k$,
\[\mathbb{E} f\left(S_n/\sqrt{n}\right)\to \int_{-\infty}^\infty f(t)
\frac{\mathrm{e}^{-t^2/2}}{\sqrt{2\uppi}}\,\mathrm{d}t
\quad\mbox{as $n\to\infty$, for every $f\in\mathrm{b}
\mathcal{C}(\mathbb{R})$.}\]
\end{document}  
