% !TEX TS-program = pdflatexmk
\documentclass{article}
\usepackage[margin=1in]{geometry} 
\usepackage[parfill]{parskip}% Begin paragraphs with an empty line, no indent
\usepackage{booktabs}
\usepackage{graphicx}
\usepackage{xcolor}
\usepackage{lipsum}
\usepackage{microtype}
%SetFonts
% libertine+newtxmath
%\usepackage[full]{textcomp}
\usepackage[sb]{libertine}
\usepackage[T1]{fontenc}
%\usepackage{cabin}
\usepackage[varqu,varl]{zi4}% inconsolata
\usepackage{amsmath,amsthm}
\usepackage[libertine,bigdelims,vvarbb]{newtxmath}
\usepackage[cal=boondoxo]{mathalfa}
\useosf
\pagestyle{empty}
\def\Pr{\ensuremath{\mathbb{P}}}
\def\d{\mathrm{d}}
\begin{document}
\thispagestyle{empty}
\begin{description}
\item[Package name:] {\tt libertine} (LinuxLibertine)
\item[Derived from:] Original work, based on nineteenth century book faces
\item[Weights and shapes:]  \{m, sb, b\}, \{n, it\}. (Uses {\tt z} internally in place of {\tt sb}.)
\item[Features:]{\color{white}x}\\[-15pt]  
\begin{itemize}
\item loads Biolinum as sans serif, Libertine Mono as {\tt tt}, the latter being very wide ($6.7$\% wider than {\tt courier}) and based on Libertine glyph shapes;
\item
full set of f-ligatures;
\item \textsc{Small Caps} in all weights and shapes;
\item lining figures, both tabular $0123456789$ and proportional {\fontfamily{LinuxLibertineT-LF}\selectfont 0123456789};
\item oldstyle figures, both tabular \tabularnums{0123456789} and proportional \proportionalnums{0123456789};
\item superior figures \textsu{0123456789}; to use them for footnote markers, call the {\tt superiors} package, as below;
%\item full set of {\tt textcomp} glyphs; 
%\item \verb|\textcircled| macro: Eg, \verb|\textcircled{A}| gives \textcircled{A}. (Must load {\tt textcomp} with {\tt full} option.)
%\item Tall ascenders, overarching f, calligraphic appearance. 
\end{itemize}
\item[Typical invocation:]{\color{white}x}
\begin{verbatim}
\usepackage{textcomp}
\usepackage[sb]{libertine} 
\usepackage[varqu,varl]{zi4}% inconsolata
\usepackage[libertine,bigdelims,vvarbb]{newtxmath} % bb from STIX
\usepackage[cal=boondoxo]{mathalfa} % mathcal
\useosf % osf for text, not math
\usepackage[supstfm=libertinesups,%
  supscaled=1.2,%
  raised=-.13em]{superiors}
\end{verbatim}
\item[Example using this preamble:]{\color{white}x}\\[6pt]
\lipsum[1]
%\textit{\lipsum[2]}
\def\Pr{\ensuremath{\mathbb{P}}}
\def\d{\mathrm{d}}
%\thispagestyle{empty}
The typeset math below follows the ISO recommendations that only variables
be set in italic. Note the use of upright shapes for $\d$, $\mathrm{e}$
and $\uppi$. (The first two are entered as \verb|\mathrm{d}| and
\verb|\mathrm{e}|, and in fonts derived from {\tt mtpro2} or {\tt newtxmath},
 the latter is entered as \verb|\uppi|.)

\textbf{Simplest form of the \textit{Central Limit Theorem}:} \textit{Let
$X_1$, $X_2,\cdots$ be a sequence of iid random variables with mean~$0$ 
and variance $1$ on a probability space $(\Omega,\mathcal{F},\Pr)$. Then}
\[\Pr\left(\frac{X_1+\cdots+X_n}{\sqrt{n}}\le y\right)\to\mathfrak{N}(y)\coloneq
\int_{-\infty}^y \frac{\mathrm{e}^{-t^2/2}}{\sqrt{2\uppi}}\,
\mathrm{d}t\quad\mbox{as $n\to\infty$,}\]
\textit{or, equivalently, letting} $S_n\coloneq\sum_1^n X_k$,
\[\mathbb{E} f\left(S_n/\sqrt{n}\right)\to \int_{-\infty}^\infty f(t)
\frac{\mathrm{e}^{-t^2/2}}{\sqrt{2\uppi}}\,\mathrm{d}t
\quad\mbox{as $n\to\infty$, for every $f\in\mathrm{b}
\mathcal{C}(\mathbb{R})$.}\]
\end{description}
\end{document}  
